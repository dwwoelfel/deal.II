\documentclass{article}

\usepackage{amsmath}
\usepackage{amssymb}

\title{Documentation of step-42, An obstacle problem for elasto-plastic material behavior in three dimensions}
\author{Joerg Frohne}
\date{Juni, 2012}

\begin{document}

\section{Introduction}

This example is an extension of step-41, considering a contact problem with an
elasto-plastic material behavior with isotropic hardening in three dimensions.
That means that we have to take care of an additional nonlinearity: the
material behavior. Since we consider a three dimensional problem here, a
separate difference to step-41 is that the contact area is at the boundary of
the deformable body now, rather than in the interior. Finally, compared to
step-41, we also have to deal with
hanging nodes because of the adaptive mesh in both the handling of the linear
system as well as of the inequality constraints; in the latter case, we will
have to deal with prioritizing whether the constraints from the hanging nodes
or from the inequalities are more important.

Since you can very easily reach a few million degrees of freedom in three
dimensions, even with adaptive mesh refinement, we decided to use Trilinos and
p4est to run our code in parallel, building on the framework of step-40 for
the parallelization.\\

\begin{huge}
{distributed}
\end{huge}


\section{Classical formulation}

The classical formulation of the problem possesses the following form:
\begin{align*}
 \varepsilon(u) &= A\sigma + \lambda & &\quad\text{in } \Omega,\\
 \lambda(\tau - \sigma) &\geq 0\quad\forall\tau\text{ with
 }\mathcal{F}(\tau)\leq 0 & &\quad\text{in } \Omega,\\
 -\textrm{\textrm{div}}\ \sigma &= f & &\quad\text{in } \Omega,\\
 u(\mathbf x) &= 0 & &\quad\text{on }\Gamma_D,\\
 \sigma_t(u) &= 0,\quad\sigma_n(u)\leq 0 & &\quad\text{on }\Gamma_C,\\
\sigma_n(u)(u_n - g) &= 0,\quad u_n(\mathbf x) - g(\mathbf x) \leq 0 & &\quad\text{on } \Gamma_C
\end{align*}
with $u\in H^2(\Omega),\Omega\subset\mathbb{R}^3$.  The vector valued
function $u$ denotes the displacement in the deformable body. The first two lines describe the
elasto-plastic material behavior. Therein the equation shows the
strain of the deformation $\varepsilon (u)$ as the additive decomposition of the
elastic part $A\sigma$ and the plastic part $\lambda$. $A$ is defined as the compliance tensor of fourth order which contains some material constants and $\sigma$ as the
symmetric stress tensor of second order. So we have to consider the inequality in the second
row component-by-component and in a pointwise sense. Furthermore we have to
distinguish two cases.\\
The continuous and convex function $\mathcal{F}$ denotes the von Mises flow function
$$\mathcal{F}(\tau) = \vert\tau^D\vert - \sigma_0¸\quad\text{with}\quad \tau^D
= \tau - \dfrac{1}{3}tr(\tau)I,$$
$\sigma_0$ as yield stress and $\vert .\vert$ as the frobenius norm. If there
are no plastic deformations in a particular point - that is $\lambda=0$ - this yields $\vert\sigma^D\vert <
\sigma_0$ and otherwise if $\lambda > 0$ it follows that $\vert\sigma^D\vert = \sigma_0$.
That means if the stress is smaller than the yield stress there are only elastic
deformations in that point.\\
To consider it the other way around if the deviator stress $\sigma^D$ is in a
norm bigger than the yield stress then $\sigma^D$ has to be projected back to the yield surface and there are plastic deformations which means $\lambda$
would be positiv for that particular point. We refer that the stresses are
computed by Hooke's law for isotorpic materials. You can find the description at the end of section 3. Else if the norm of the deviator stress tensor is smaller or equal the yield stress then $\lambda$ is zero and there are no plastic deformations in
that point.\\
There the index $D$ denotes the deviator part of for example the stress where
$tr(.)$ is the trace of a tensor. The definition shows an additive decomposition
of the stress $\sigma$ into a hydrostatic part (or volumetric part) $\dfrac{1}{3}tr(\tau)I$ and the deviator
part $\sigma^D$. For metal the deviator stress composes the main indicator for
plastic deformations.\\
The third equation is called equilibrium condition with a force of volume
density $f$ which we will neglect in our example.
The boundary of $\Omega$ separates as follows $\Gamma=\Gamma_D\bigcup\Gamma_C$ and $\Gamma_D\bigcap\Gamma_C=\emptyset$.
At the boundary $\Gamma_D$ we have zero Dirichlet conditions. $\Gamma_C$ denotes the potential contact boundary.\\
The last two lines decribe the so-called Signorini contact conditions. If there is no contact the normal  stress
$$ \sigma_n =  \sigma n\cdot n$$
is zero with the outward normal $n$. If there is contact ($u_n = g$) the tangential stress $\sigma_t = \sigma\cdot n - \sigma_n n$
vanishes, because we consider a frictionless situation and the normal stress is
negative. The gap $g$ comes with the start configuration of the obstacle and the
deformable body.

\section{Derivation of the variational inequality}

As a starting point to derive the equations above, let us imagine that we want
to minimise an energy functional:
$$E(\tau) := \dfrac{1}{2}\int\limits_{\Omega}\tau A \tau d\tau,\quad \tau\in \Pi W^{\textrm{div}}$$
with
$$W^{\textrm{div}}:=\lbrace \tau\in
L^2(\Omega,\mathbb{R}^{\textrm{dim}\times\textrm{dim}}_{\textrm{sym}}):\textrm{div}(\tau)\in L^2(\Omega,\mathbb{R}^{\textrm{dim}})\rbrace$$ and
$$\Pi \Sigma:=\lbrace \tau\in \Sigma, \mathcal{F}(\tau)\leq 0\rbrace$$
as the set of admissible stresses which is defined
by a continious, convex flow function $\mathcal{F}$.

With the goal of deriving the dual formulation of the minimisation
problem, we define a lagrange function:
$$L(\tau,\varphi) := E(\tau) + (\varphi, \textrm{div}(\tau)),\quad \lbrace\tau,\varphi\rbrace\in\Pi W^{\textrm{div}}\times V^+$$
with
$$V^+ := \lbrace u\in V: u_n\leq g \text{ on } \Gamma_C \rbrace$$
$$V:=\left[ H_0^1 \right]^{\textrm{dim}}:=\lbrace u\in \left[H^1(\Omega)\right]^{\textrm{dim}}: u
= 0 \text{ on } \Gamma_D\rbrace$$
By building the Fr\'echet derivatives of $L$ for both components we obtain the
dual formulation for the stationary case which is known as \textbf{Hencky-Type-Model}:\\
Find a pair $\lbrace\sigma,u\rbrace\in \Pi W\times V^+$ with
$$\left(A\sigma,\tau - \sigma\right) + \left(u, \textrm{div}(\tau) - \textrm{div}(\sigma)\right) \geq 0,\quad \forall \tau\in \Pi W^{\textrm{div}}$$
$$-\left(\textrm{div}(\sigma),\varphi - u\right) \geq 0,\quad \forall \varphi\in V^+.$$
By integrating by parts and multiplying the first inequality by the elastic
tensor $C=A^{-1}$ we achieve the primal-mixed version of our problem:\\
Find a pair $\lbrace\sigma,u\rbrace\in \Pi W\times V^+$ with
$$\left(\sigma,\tau - \sigma\right) - \left(C\varepsilon(u), \tau - \sigma\right) \geq 0,\quad \forall \tau\in \Pi W$$
$$\left(\sigma,\varepsilon(\varphi) - \varepsilon(u)\right) \geq 0,\quad \forall \varphi\in V^+.$$
Therein $\varepsilon$ denotes the linearised deformation tensor with $\varepsilon(u) := \dfrac{1}{2}\left(\nabla u + \nabla u^T\right)$ for small deformations.\\
Most materials - especially metals - have the property that they show some hardening effects during the forming process.
There are different constitutive laws to describe those material behaviors. The
simplest one is called linear isotropic hardening described by the flow function
$\mathcal{F}(\tau,\eta) = \vert\tau^D\vert - (\sigma_0 + \gamma\eta)$ where
$\eta$ is the norm of the plastic strain $\eta = \vert \varepsilon -
A\sigma\vert$.
It can be considered by establishing an additional term in our primal-mixed formulation:\\
Find a pair $\lbrace(\sigma,\xi),u\rbrace\in \Pi (W\times L^2(\Omega,\mathbb{R}))\times V^+$ with
$$\left(\sigma,\tau - \sigma\right) - \left(C\varepsilon(u), \tau - \sigma\right) + \gamma\left( \xi, \eta - \xi\right) \geq 0,\quad \forall (\tau,\eta)\in \Pi (W,L^2(\Omega,\mathbb{R}))$$
$$\left(\sigma,\varepsilon(\varphi) - \varepsilon(u)\right) \geq 0,\quad \forall \varphi\in V^+,$$
with the hardening parameter $\gamma > 0$.\\
Now we want to derive a primal problem which only depends on the displacement $u$. For that purpose we
set $\eta = \xi$ and eliminate the stress $\sigma$ by applying the projection
theorem (see Grossmann, Roos: Numerical Treatment of Partial Differential
Equations, Springer-Verlag Berlin Heidelberg, 2007 and Frohne: FEM-Simulation
der Umformtechnik metallischer Oberflächen im Mikrokosmos, Ph.D. thesis,
University of Siegen, Germany, 2011) on\\
$$\left(\sigma - C\varepsilon(u), \tau - \sigma\right) \geq 0,\quad \forall \tau\in \Pi W,$$
which yields with the second inequality:\\
Find the displacement $u\in V^+$ with
$$\left(P_{\Pi}(C\varepsilon(u)),\varepsilon(\varphi) - \varepsilon(u)\right) \geq 0,\quad \forall \varphi\in V^+,$$
with the projection:
$$P_{\Pi}(\tau):=\begin{cases}
			\tau, & \text{if }\vert\tau^D\vert \leq \sigma_0 +  \gamma\xi,\\
			\hat\alpha\dfrac{\tau^D}{\vert\tau^D\vert} + \dfrac{1}{3}tr(\tau), & \text{if }\vert\tau^D\vert > \sigma_0 +  \gamma\xi,
			\end{cases}$$
with the radius
$$\hat\alpha := \sigma_0 + \gamma\xi .$$
With the relation $\xi = \vert\varepsilon(u) - A\sigma\vert$ it is possible to eliminate $\xi$ inside the projection $P_{\Pi}$:\\
$$P_{\Pi}(\tau):=\begin{cases}
			\tau, & \text{if }\vert\tau^D\vert \leq \sigma_0,\\
			\alpha\dfrac{\tau^D}{\vert\tau^D\vert} + \dfrac{1}{3}tr(\tau), & \text{if }\vert\tau^D\vert > \sigma_0,
			\end{cases}$$
$$\alpha := \sigma_0 + \dfrac{\gamma}{2\mu+\gamma}\left(\vert\tau^D\vert - \sigma_0\right) ,$$
with a further material parameter $\mu>0$ called shear modulus. We refer that
this only possible for isotropic plasticity.\\
So what we do is to calculate the stresses by using Hooke's law for linear elastic,  isotropic materials
$$\sigma = C \varepsilon(u) = 2\mu \varepsilon^D(u) + \kappa tr(\varepsilon(u))I = \left[2\mu\left(\mathbb{I} -\dfrac{1}{3} I\otimes I\right) + \kappa I\otimes I\right]\varepsilon(u)$$
with the material parameter $\kappa>0$ (bulk modulus). The variables $I$ and
$\mathbb{I}$ denote the identity tensors of second and forth order. In that
notation $2\mu \varepsilon^D(u)$ is the deviatoric part and $\kappa
tr(\varepsilon(u))$ the volumetric part of the stress tensor.\\
In the next step we test in a pointwise sense where the deviator part of the
stress in a norm is bigger than the yield stress. If there are such points we
project the deviator stress in those points back to the yield surface. Methods of this kind are called projections algorithm or radial-return-algorithm.\\
Now we have a primal formulation of our elasto-plastic contact problem which only depends on the displacement $u$.
It consists of a nonlinear variational inequality and has a unique solution as
it satisfies the theorem of Lions and Stampaccia. A proof can be found in
Rodrigues: Obstacle Problems in Mathematical Physics, North-Holland, Amsterdam,
1987.\\
To handle the nonlinearity of the constitutive law we use a Newton method and to deal with the contact we apply an
active set method like in step-41. To be more concrete we combine both methods to an inexact semi smooth Newton
method - inexact since we use an iterative solver for the linearised problems in each Newton step.

\section{Linearisation of the constitutive law for the Newton method}

For the Newton method we have to linearise the following semi-linearform
$$a(\psi;\varphi) := \left(P_{\Pi}(C\varepsilon(\varphi)),\varepsilon(\varphi)\right).$$
Because we have to find the solution $u$ in the convex set $V^+$, we have to
apply an SQP-method (SQP: sequential quadratic programming). That means we have
to solve a minimisation problem for a known $u^i$ in every SQP-step of the form
\begin{eqnarray*}
 & & a(u^{i};u^{i+1} - u^i) + \dfrac{1}{2}a'(u^i;u^{i+1} - u^i,u^{i+1} - u^i)\\
 &=&  a(u^i;u^{i+1}) -  a(u^i;u^i) +\\
 & & \dfrac{1}{2}\left( a'(u^i;u^{i+1},u^{i+1}) - 2a'(u^i;u^i,u^{i+1}) - a'(u^i;u^i,u^i)\right)\\
 &\rightarrow& \textrm{min},\quad u^{i+1}\in V^+.
\end{eqnarray*}
Neglecting the constant terms $ a(u^i;u^i)$ and $ a'(u^i;u^i,u^i)$ we obtain the
following minimisation problem $$\dfrac{1}{2} a'(u^i;u^{i+1},u^{i+1}) - F(u^i)\rightarrow \textrm{min},\quad u^{i+1}\in V^+$$ with
$$F(\varphi) := \left(a'(\varphi;\varphi,u^{i+1}) -  a(\varphi;u^{i+1}) \right).$$
In the case of our constitutive law the Fr\'echet derivative of the
semi-linearform $a(.;.)$ at the point $u^i$ is

$$a'(u^i;\psi,\varphi) =
(I(x)\varepsilon(\psi),\varepsilon(\varphi)),\quad x\in\Omega,$$ $$
I(x) := \begin{cases}
2\mu\left(\mathbb{I}  - \dfrac{1}{3} I\otimes I\right) + \kappa I\otimes I, &
\quad \vert \tau^D \vert \leq \sigma_0\\
\dfrac{\alpha}{\vert\tau^D\vert}2\mu\left(\mathbb{I}  - \dfrac{1}{3} I\otimes I 
- \dfrac{\tau^D\otimes\tau^D}{\vert\tau^D\vert}\right) + \kappa I\otimes I,
&\quad \vert \tau^D \vert > \sigma_0
\end{cases}
$$
with
$$\tau^D :=  C\varepsilon^D(u^i).$$
Remark that $a(.;.)$ is not differentiable in the common sense but it is
slantly differentiable like the function for the contact problem and again we refer to
Hintermueller, Ito, Kunisch: The primal-dual active set strategy as a semismooth newton method, SIAM J. OPTIM., 2003, Vol. 13, No. 3, pp. 865-888.
Again the first case is for elastic and the second for plastic deformation.

\section{Formulation as a saddle point problem}

Just as in step-41 we compose a saddle point problem out of the minimisation
problem. Again we do so to gain a formulation that allows us to solve a linear
system of equations finally.\\
We introduce a Lagrange multiplier $\lambda$ and the convex cone $K\subset W'$,
$W'$ dual space of the trace space $W$ of $V$ restricted to $\Gamma_C$,
$$K:=\{\mu\in W':\mu_T = 0,\quad\langle\mu n,v\rangle_{\Gamma_C}\geq 0,\quad
\forall v\in W, v \ge 0\text{ on }\Gamma_C \}$$
of Lagrange multipliers, where $\langle\cdot,\cdot\rangle$
denotes the duality pairing, i.e. a boundary integral, between $W'$ and $W$.
Intuitively, $K$ is the cone of all "non-positive functions", except that $ K\subset
\left( \left[ H_0^{\frac{1}{2}} \right]^{\textrm{dim}} \right)' $ and so contains other
objects besides regular functions as well. This yields:\\

\noindent
\textit{Find $u\in V$ and $\lambda\in K$ such that}
\begin{align*}
 \hat{a}(u,v) + b(v,\lambda) &= f(v),\quad &&v\in V\\
 b(u,\mu - \lambda) &\leq \langle g,(\mu -
 \lambda)n\rangle_{\Gamma_C},\quad&&\mu\in K,
\end{align*}
\textit{with}
\begin{align*}
 \hat{a}(u,v) &:= a'(u^i;u,v)\\
 b(u,\mu) &:= \langle un,\mu n\rangle_{\Gamma_C},\quad &&u\in V,\quad\mu\in W'.
\end{align*}
As in the section before $u^i$ denotes the linearization point for the
semi-linearform $a(.;.)$. In contrast to step-41 we directly consider $\lambda$
as the additional, positive force $\sigma(u)n$ that the obstacle
exerts on the boundary $\Gamma_C$ of the body.\\

\noindent
The existence and uniqueness of the analytical solution $(u,\lambda)\in V\times
K$ of this saddle point problem has been stated in Glowinski, Lions and Tr\'{e}moli\`{e}res: Numerical
Analysis of Variational Inequalities, North-Holland, 1981.\\

\noindent
NOTE: In this example as well as in the further documentation we make the
assumption that the normal vector $n$ equals to $(0,0,1)$. This comes up with
the starting condition of our deformable body.

\section{Active Set methods to solve the saddle point problem}

The linearized problem is essentially like a pure elastic problem with contact like
in step-41. The only difference consists in the fact that the contact area
is at the boundary instead of in the domain. But this has no further consequence
so that we refer to the documentation of step-41 with the only hint that
$\mathcal{S}$ containts all the vertices at the contact boundary $\Gamma_C$ this
time.

\section{The primal-dual active set algorithm combined with the inexact semi smooth
Newton method}

Now we describe an algorithm that combines the Newton-method, which we use for
the nonlinear constitutive law, with the semismooth Newton method for the contact. It
sums up the results of the sections before and works as follows:
\begin{itemize}
 \item[(0)] Initialize $\mathcal{A}_k$ and $\mathcal{F}_k$, such that $\mathcal{S} = \mathcal{A}_k \cup \mathcal{F}_k$ and $\mathcal{A}_k \cap \mathcal{F}_k = \emptyset$ and set $k = 1$.
 \item[(1)] Assemble the Newton matrix $a'(U^k;\varphi_i,\varphi_j)$ and the right-hand-side $F(U^k)$.
 \item[(2)] Find the primal-dual pair $(U^k,\Lambda^k)$ that satisfies
 \begin{align*}
 AU^k + B\Lambda^k & = F, &\\
 \left[B^TU^k\right]_i & = G_i & \forall i\in\mathcal{A}_k,\\
 \Lambda^k_i & = 0 & \forall i\in\mathcal{F}_k.
 \end{align*}
 \item[(3)] Define the new active and inactive sets by
 $$\mathcal{A}_{k+1}:=\lbrace i\in\mathcal{S}:\Lambda^k_i +
 c\left(\left[B^TU^k\right]_i - G_i\right) > 0\rbrace,$$
 $$\mathcal{F}_{k+1}:=\lbrace i\in\mathcal{S}:\Lambda^k_i +
 c\left(\left[B^TU^k\right]_i - G_i\right) \leq 0\rbrace.$$
 \item[(4)] If $\mathcal{A}_{k+1} = \mathcal{A}_k$ and $\vert
 F\left(U^{k+1}\right) \vert < \delta$ then stop, else set $k=k+1$ and go to
 step (1).
\end{itemize}
\noindent
The mass matrix $B\in\mathbb{R}^{n\times m}$, $n>m$, is quadratic in our
situation since $\Lambda^k$ is only defined on $\Gamma_C$:
$$B_{ij} = \begin{cases}
\int\limits_{\Gamma_C}\varphi_i^2(x)dx, & \text{if}\quad i=j\\
0, & \text{if}\quad i\neq j.
\end{cases}$$
So $m$ denotes the size of $\Lambda^k$ and $i$ a degree of freedom. In our
programm we use the structure of a quadratic sparse for $B\in\mathbb{R}^{n\times
n}$ and the length of $\Lambda^k$ is $n$ with $\Lambda^k_i = 0$ for $i>m$.
The vector $G$ is defined by a suitable approximation $g_h$ of the gap $g$
$$G_i = \begin{cases}
\int\limits_{\Gamma_C}g_h(x)\varphi_i(x)dx, & \text{if}\quad i\leq m\\
0, & \text{if}\quad i>m.
\end{cases}$$\\
Compared to step-41, step (1) is added but it should be clear
from the sections above that we only linearize the problem. In step (2) we have to solve a linear
system of equations again. And now the solution has to fulfill two stopping
criteria. $\mathcal{A}_{k+1} = \mathcal{A}_k$ makes sure that the contact zones are iterated out and the second ensures an accurate enough residual which means
that the plastic zones are also iterated out.\\
The idea of this method can also be found in Brunssen, Schmid, Schaefer,
Wohlmuth: A fast and robust iterative solver for nonlinear contact problems
using a primal-dual active set strategy and algebraic multigrid, Int. J. Numer.
Meth. Engng, 2007, 69, pp. 524-543.

\section{Implementation}

\end{document}
