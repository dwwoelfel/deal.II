\documentclass{article}
\usepackage{amssymb,amsmath}
\makeatletter
\newcommand{\rmnum}[1]{\romannumeral #1}
\newcommand{\Rmnum}[1]{\expandafter\@slowromancap\romannumeral #1@}
\makeatother
\begin{document}

What is new in this example:
\begin{enumerate}
\item Solve multigroup neutron diffusion problem with multiple different meshes
\item Solve an eigenvalue problem
\item Setting up complicated material properties for nuclear fuel assemblies
\end{enumerate}

\subsection{Introduction}

In this example, we intend to solve the multigroup diffusion approximation of
the neutron transport equation. Essentially, the way to view this is as follows: In a
nuclear reactor, neutrons are speeding around at different energies, get
absorbed or scattered, or start a new fission
event. If viewed at long enough length scales, the movement of neutrons can be
considered a diffusion process. 

A mathematical description of this would group neutrons into energy bins, and
consider the balance equations for the neutron fluxes in each of these
bins, or energy groups. The scattering, absorption, and fission events would
then be operators within the diffusion equation describing the neutron
fluxes. Assume we have energy groups $g=1,\ldots,G$, where by convention we
assume that the neutrons with the highest energy are in group 1 and those with
the lowest energy in group $G$. Then the neutron flux of each group satisfies the
following equations:
\begin{eqnarray*}
\frac 1{v_g}\frac{\partial \phi_g(x,t)}{\partial t}
&=&
\nabla \cdot(D_g(x) \nabla \phi_g(x,t))
-
\Sigma_{r,g}(x)\phi_g(x,t)
\\
&& \qquad
+
\chi_g\sum_{g'=1}^G\nu\Sigma_{f,g'}(x)\phi_{g'}(x,t)
+
\sum_{g'\ne g}\Sigma_{s,g'\to g}(x)\phi_{g'}(x,t)
+
s_{\mathrm{ext},g}(x,t)
\end{eqnarray*}
augmented by appropriate boundary conditions. Here, $v_g$ is the velocity of
neutrons within group $g$. In other words, the change in
time in flux of neutrons in group $g$ is governed by the following
processes:
\begin{itemize}
\item Diffusion $\nabla \cdot(D_g(x) \nabla \phi_g(x,t))$. Here, $D_g$ is the
  (spatially variable) diffusion coefficient. 
\item Absorption $\Sigma_{r,g}(x)\phi_g(x,t)$ (note the
  negative sign). The coefficient $\Sigma_{r,g}$ is called the \textit{removal
  cross section}.
\item Nuclear fission $\chi_g\sum_{g'=1}^G\nu\Sigma_{f,g'}(x)\phi_{g'}(x,t)$. 
  The production of neutrons of energy $g$ is
  proportional to the flux of neutrons of energy $g'$ times the
  probability $\Sigma_{f,g'}$ that neutrons of energy $g'$ cause a fission
  event times the number $\nu$ of neutrons produced in each fission event
  times the probability that a neutron produced in this event has energy
  $g$. $\nu\Sigma_{f,g'}$ is called the \textit{fission cross section} and
  $\chi_g$ the \textit{fission spectrum}. We will denote the term
  $\chi_g\nu\Sigma_{f,g'}$ as the \textit{fission distribution cross
    section} in the program.
\item Scattering $\sum_{g'\ne g}\Sigma_{s,g'\to g}(x)\phi_{g'}(x,t)$ 
  of neutrons of energy $g'$ producing neutrons
  of energy $g$. $\Sigma_{s,g'\to g}$ is called the \textit{scattering cross
    section}. The case of elastic, in-group scattering $g'=g$ exists, too, but
  we subsume this into the removal cross section. The case $g'<g$ is called
  down-scattering, since a neutron loses energy in such an event. On the
  other hand, $g'>g$ corresponds to up-scattering: a neutron gains energy in
  a scattering event from the thermal motion of the atoms surrounding it;
  up-scattering is therefore only an important process for neutrons with
  kinetic energies that are already on the same order as the thermal kinetic
  energy (i.e. in the sub $eV$ range).
\item An extraneous source $s_{\mathrm{ext},g}$.
\end{itemize}

For realistic simulations in reactor analysis, one may want to split the
continuous spectrum of neutron energies into many energy groups, often up to
100. However, if neutron energy spectra are known well enough for some type of
reactor (for example Pressurized Water Reactors, PWR), it is possible to obtain
satisfactory results with only 2 energy groups. 

In the program shown in this tutorial program, we provide the structure to
compute with as many energy groups as desired. However, to keep computing
times moderate and in order to avoid tabulating hundreds of coefficients, we
only provide the coefficients for above equations for a two-group simulation,
i.e. $g=1,2$. We do, however, consider a realistic situation by assuming that
the coefficients are not constant, but rather depend on the materials that are
assembled into reactor fuel assemblies in rather complicated ways (see
below). 


\subsection{The eigenvalue problem}

If we consider all energy groups at once, we may write above equations in the
following operator form:
\begin{equation}
\frac 1v \frac{\partial \phi}{\partial t}
=
-L\phi
+
F\phi
+
X\phi
+
s_{\mathrm{ext}},
\end{equation}
where $L,F,X$ are sinking, fission, and scattering operators,
respectively. $L$ here includes both the diffusion and removal terms. Note
that $L$ is symmetric, whereas $F$ and $X$ are not.

It is well known that this equation admits a stable solution if all
eigenvalues of the operator $-L+F+X$ are negative. This can be readily seen by
multiplying the equation by $\phi$ and integrating over the domain, leading to
\begin{equation}
  \frac 1{2v} \frac{\partial}{\partial t}  \|\phi\|^2 = ((-L+F+X)\phi,\phi).
\end{equation}
Stability means that the solution does not grow, i.e. we want the left hand
side to be less than zero, which is the case if the eigenvalues of the
operator on the right are all negative. For obvious reasons, it is
not very desirable if a nuclear reactor produces neutron fluxes that grow
exponentially, so eigenvalue analyses are the bread-and-butter of nuclear
engineers. The main point of the program is therefore to consider the
eigenvalue problem
\begin{equation}
  (L-F-X) \phi = \lambda \phi,
\end{equation}
where we want to make sure that all eigenvalues are positive. Note that $L$,
being the diffusion operator plus the absorption (removal), is positive
definite; the condition that all eigenvalues are positive therefore means that
we want to make sure that fission and inter-group scattering are weak enough
to not shift the spectrum into the negative.

In nuclear engineering, one typically looks at a slightly different
formulation of the eigenvalue problem. To this end, we do not just multiply
with $\phi$ and integrate, but rather multiply with $\phi(L-X)^{-1}$. We then
get the following evolution equation:
\begin{equation}
  \frac 1{2v} \frac{\partial}{\partial t}  \|\phi\|^2_{(L-X)^{-1}} = ((L-X)^{-1}(-L+F+X)\phi,\phi).
\end{equation}
Stability is the guaranteed if the eigenvalues of the following problem are
all negative:
\begin{equation}
  (L-X)^{-1}(-L+F+X)\phi = \lambda_F \phi,
\end{equation}
which is equivalent to the eigenvalue problem
\begin{equation}
  (L-X)\phi = \frac 1{\lambda_F+1} F \phi.
\end{equation}
The typical formulation in nuclear engineering is to write this as
\begin{equation}
  (L-X) \phi = \frac 1{k_{\mathrm{eff}}} F \phi,
\end{equation}
where $k_{\mathrm{eff}}=\frac 1{\lambda^F+1}$.
Intuitively, $k_{\mathrm{eff}}$ is something like the multiplication
factor for neutrons per typical time scale and should be less than or equal to
one for stable operation of a reactor: if it is less than one, the chain reaction will
die down, whereas nuclear bombs for example have a $k$-eigenvalue larger than
one. A stable reactor should have $k_{\mathrm{eff}}=1$.

[For those who wonder how this can be achieved in practice without
inadvertently getting slightly larger than one and triggering a nuclear bomb:
first, fission processes happen on different time scales. While most neutrons
are releases very quickly after a fission event, a small number of neutrons
are only released by daughter nuclei after several further decays, up to 10-60
seconds after the fission was initiated. If one is therefore slightly beyond
$k_{\mathrm{eff}}=1$, one therefore has many seconds to react until all the
neutrons created in fission re-enter the fission cycle. Nevertheless, control
rods in nuclear reactors absorbing neutrons -- and therefore reducing
$k_{\mathrm{eff}}$ -- are designed in such a way that they are all the way in
the reactor in at most 2 seconds. 

One therefore has on the order of 10-60 seconds to regulate the nuclear reaction
if $k_{\mathrm{eff}}$ should be larger than one for some time, as indicated by
a growing neutron flux. Regulation can be achieved by continuously monitoring
the neutron flux, and if necessary increase or reduce neutron flux by moving
neutron-absorbing control rods a few millimeters into or out of the
reactor. On a longer scale, the water cooling the reactor contains boron, a
good neutron absorber. Every few hours, boron concentrations are adjusted by
adding boron or diluting the coolant.

Finally, some of the absorption and scattering reactions have some
stability built in; for example, higher neutron fluxes result in locally
higher temperatures, which lowers the density of water and therefore reduces
the number of scatterers that are necessary to moderate neutrons from high to
low energies before they can start fission events themselves.]

In this tutorial program, we solve above $k$-eigenvalue problem for two energy
groups, and we are looking for the largest multiplication factor
$k_{\mathrm{eff}}$, which is proportional to the inverse of the minimum
eigenvalue plus one. To solve the eigenvalue problem, we generally
use a modified version of the \emph{inverse power method}. The algorithm looks
like this:

\begin{enumerate}
\item Initialize $\phi_g$ and $k_{\mathrm{eff}}$ with $\phi_g^{(0)}$
  and $k_{\mathrm{eff}}^{(0)}$ and let $n=1$. 

\item Define the so-called \textit{fission source} by
  \begin{equation}
    s_f^{(n-1)}(x)
    =
    \frac{1}{k_{\mathrm{eff}}^{(n-1)}}
    \sum_{g'=1}^G\nu\Sigma_{f,g'}(x)\phi_{g'}^{(n-1)}(x).
  \end{equation}

\item Solve for all group fluxes $\phi_g,g=1,\ldots,G$ using
  \begin{equation}
    -\nabla \cdot D_g\nabla \phi_g^{(n)}
    +
    \Sigma_{r,g}\phi_g^{(n)}
    = 
    \chi_g s_f^{(n-1)}
    + 
    \sum_{g'< g} \Sigma_{s,g'\to g} \phi_{g'}^{(n)}
    +
    \sum_{g'> g}\Sigma_{s,g'\to g}\phi_{g'}^{(n-1)}.
  \end{equation}

\item Update
  \begin{equation}
    k_{\mathrm{eff}}^{(n)}
    =
    \sum_{g'=1}^G
    \int_{\Omega}\nu\Sigma_{f,g'}(x)
    \phi_{g'}^{(n)}(x)dx.
  \end{equation}

\item Compare $k_{\mathrm{eff}}^{(n)}$ with $k_{\mathrm{eff}}^{(n-1)}$.
  If the change greater than a prescribed tolerance then set $n=n+1$ repeat
  the iteration starting at step 2, otherwise end the iteration.
\end{enumerate}

Note that in this scheme, we do not solve group fluxes exactly in each power
iteration, but rather consider previously compute $\phi_{g'}^{(n)}$ only for
down-scattering events $g'<g$. Up-scattering is only treated by using old
iterators $\phi_{g'}^{(n-1)}$, in essence assuming that the scattering
operator is triangular. This is physically motivated since up-scattering does
not play a too important role in neutron scattering. In addition, practices
shows that the inverse power iteration is stable even using this
simplification.

Note also that one can use lots of extrapolation techniques to accelerate the
power iteration laid out above. However, none of these are implemented in this
example.


\subsection{Meshes and mesh refinement}

One may wonder whether it is appropriate to solve for the solutions of the
individual energy group equations on the same meshes. The question boils down
to this: will $\phi_g$ and $\phi_{g'}$ have similar smoothness properties? If
this is the case, then it is appropriate to use the same mesh for the two; a
typical application could be chemical combustion, where typically the
concentrations of all or most chemical species change rapidly within the flame
front. As it turns out, and as will be apparent by looking at the
graphs shown in the results section of this tutorial program, this isn't the
case here, however: since the diffusion coefficient is different for different
energy groups, fast neutrons (in bins with a small group number $g$) have a very
smooth flux function, whereas slow neutrons (in bins with a large group
number) are much more affected by the local material properties and have a
correspondingly rough solution if the coefficient are rough as in the case we
compute here. Consequently, we will want to use different meshes to compute
each energy group.

This has two implications that we will have to consider: First, we need to
find a way to refine the meshes individually. Second, assembling the source
terms for the inverse power iteration, where we have to integrate solution
$\phi_{g'}^{(n)}$ defined on mesh $g'$ against the shape functions defined on
mesh $g$, becomes a much more complicated task.


\subsubsection{Mesh refinement}

We use the usual paradigm: solve on a given mesh, then evaluate an error
indicator for each cell of each mesh we have. Because it is so convenient, we
again use the \emph{a posteriori} error estimator by Kelly, Gago, Zienkiewicz
and Babuska which approximates the error per cell by integrating the jump of
the gradient of the solution along the faces of each cell. Using this, we
obtain indicators
\begin{equation}
\eta_{g,K}, \qquad g=1,2,\ldots,G,\qquad K\in{\mathbb T}_g,
\end{equation}
where ${\mathbb T}_g$ is the triangulation used in the solution of
$\phi_g$. The question is what to do with this. For one, it is clear that
refining only those cells with the highest error indicators might lead to bad
results. To understand this, it is important to realize that $\eta_{g,K}$
scales with the second derivative of $\phi_g$. In other words, if we have two
energy groups $g=1,2$ whose solutions are equally smooth but where one is
larger by a factor of 10,000, for example, then only the cells of that mesh
will be refined, whereas the mesh for the solution of small magnitude will
remain coarse. This is probably not what one wants, since we can consider both
components of the solution equally important.

In essence, we would therefore have to scale $\eta_{g,K}$ by an importance
factor $z_g$ that says how important it is to resolve $\phi_g$ to any given
accuracy. Such important factors can be computed using duality techniques
(see, for example, the step-14 tutorial program, and the
reference to the book by Bangerth and Rannacher cited there). We won't go
there, however, and simply assume that all energy groups are equally
important, and will therefore normalize the error indicators $\eta_{g,K}$ for
group $g$ by the maximum of the solution $\phi_g$. We then refine the cells
whose errors satisfy
\begin{equation}
  \frac{\eta_{g,K}}{\|\phi_g\|_\infty}
  >
  \alpha_1
  \displaystyle{\max_{\substack{1\le g\le G\\K\in {\mathbb T}_g}}
    \frac{\eta_{g,K}}{\|\phi_g\|_\infty}}
\end{equation}
and coarsen the cells where
\begin{equation}
  \frac{\eta_{g,K}}{\|\phi_g\|_\infty}
  <
  \alpha_2
  \displaystyle{\max_{\substack{1\le g\le G\\K\in {\mathbb T}_g}}
    \frac{\eta_{g,K}}{\|\phi_g\|_\infty}}.
\end{equation}
We chose $\alpha_1=0.3$ and $\alpha_2=0.01$ in the code. Note that this will,
of course, lead to different meshes for the different energy groups.

The strategy above essentially means the following: If for energy group $g$
there are many cells $K\in {\mathbb T}_g$ on which the error is large, for
example because the solution is globally very rough, then many cells will be
above the threshold. On the other hand, if there are a few cells with large
and many with small errors, for example because the solution is overall rather
smooth except at a few places, then only the few cells with large errors will
be refined. Consequently, the strategy allows for meshes that track the global
smoothness properties of the corresponding solutions rather well.


\subsubsection{Assembling terms on different meshes}

As pointed out above, the multigroup refinement strategy results in
different meshes for the different solutions $\phi_g$. So what's the problem?
In essence it goes like this: in step 3 of the eigenvalue iteration, we have
form the weak form for the equation to compute $\phi_g^{(n)}$ as usual by
multiplication with test functions $\varphi_g^i$ defined on the mesh for
energy group $g$; in the process, we have to
compute the right hand side vector that contains terms of the following form:
\begin{equation}
  F_i = \int_\Omega f(x) \varphi_g^i(x) \phi_{g'}(x) \ dx,
\end{equation}
where $f(x)$ is one of the coefficient functions $\Sigma_{s,g'\to g}$ or
$\nu\chi_g\Sigma_{f,g'}$ used in the right hand side 
of eigenvalue equation. The difficulty now is that $\phi_{g'}$ is defined on
the mesh for energy group $g'$, i.e. it can be expanded as
$\phi_{g'}(x)=\sum_j\phi_{g'}^j \varphi_{g'}^j(x)$, with basis functions
$\varphi_{g'}^j(x)$ defined on mesh $g'$. The contribution to the right hand
side can therefore be written as
\begin{equation}
  F_i = \sum_j \left\{\int_\Omega f(x) \varphi_g^i(x) \varphi_{g'}^j(x) 
  \ dx \right\} \phi_{g'}^j , 
\end{equation}
On the other hand, the test functions $\varphi_g^i(x)$ are defined on mesh
$g$. This means that we can't just split the integral $\Omega$ into integrals
over the cells of either mesh $g$ or $g'$, since the respectively other basis
functions may not be defined on these cells.

The solution to this problem lies in the fact that both the meshes for $g$ and
$g'$ are derived by adaptive refinement from a common coarse mesh. We can
therefore always find a set of cells, which we denote by ${\mathbb T}_g \cap
{\mathbb T}_{g'}$, that satisfy the following conditions:
\begin{itemize}
\item the union of the cells covers the entire domain, and
\item a cell $K \in {\mathbb T}_g \cap {\mathbb T}_{g'}$ is active on at least
  one of the two meshes.
\end{itemize}
A way to construct this set is to take each cell of coarse mesh and do the
following steps: (i) if the cell is active on either ${\mathbb T}_g$ or
${\mathbb T}_{g'}$, then add this cell to the set; (ii) otherwise, i.e. if
this cell has children on both meshes, then do step (i) for each of the
children of this cell. In fact, deal.II has a function
\texttt{GridTools::get\_finest\_common\_cells} that computes exactly this set
of cells that are active on at least one of two meshes.

With this, we can write above integral as follows:
\begin{equation}
  F_i 
  = 
  \sum_{K \in {\mathbb T}_g \cap {\mathbb T}_{g'}}
  \sum_j \left\{\int_K f(x) \varphi_g^i(x) \varphi_{g'}^j(x) 
  \ dx \right\} \phi_{g'}^j.
\end{equation}
 In the code, we
compute the right hand side in the function
\texttt{NeutronDiffusionProblem::assemble\_rhs}, where (among other things) we
loop over the set of common most refined cells, calling the function
\texttt{NeutronDiffusionProblem::assemble\_common\_cell} on each pair of
these cells.

By construction, there are now three cases to be considered:
\begin{itemize}
\item[(i)] The cell $K$ is active on both meshes, i.e. both the basis
  functions $\varphi_g^i$ as well as $\varphi_{g'}^j$ are defined on $K$.
\item[(ii)] The cell $K$ is active on mesh $g$, but not $g'$, i.e. the
  $\varphi_g^i$  are defined on $K$, whereas the $\varphi_{g'}^j$ are defined
  on children of $K$.
\item[(iii)] The cell $K$ is active on mesh $g'$, but not $g$, with opposite
  conclusions than in (ii).
\end{itemize}

To compute the right hand side above, we then need to have different code for
these three cases, as follows:
\begin{itemize}
\item[(i)] If the cell $K$ is active on both meshes, then we can directly
  evaluate the integral. In fact, we don't even have to bother with the basis
  functions $\varphi_{g'}$, since all we need is the values of $\phi_{g'}$ at
  the quadrature points. We can do this using the
  \texttt{FEValues::get\_function\_values} function. This is done directly in
  the \texttt{NeutronDiffusionProblem::assemble\_common\_cell} function.

\item[(ii)] If the cell $K$ is active on mesh $g$, but not $g'$, then the
  basis functions $\varphi_{g'}^j$ are only defined either on the children
  $K_c,0\le c<2^{\texttt{dim}}$, or on children of these children if cell $K$
  is refined more than once more on mesh $g'$.

  Let us assume for a second that $K$ is only once more refined on mesh $g'$
  than on mesh $g$. Using the fact that we use embedded finite element spaces
  where each basis function on one mesh can be written as a linear combination
  of basis functions on the next refined mesh, we can expand the restriction
  of $\phi_g^i$ to child cell $K_c$ into the basis functions defined on that
  child cell (i.e. on cells on which the basis functions $\varphi_{g'}^l$ are
  defined):
  \begin{equation}
    \phi_g^i|_{K_c} = B_c^{il} \varphi_{g'}^l|_{K_c}.
  \end{equation}
  Here, and in the following, summation over indices appearing twice is
  implied. The matrix $B_c$ is the matrix that interpolated data from a cell
  to its $c$-th child.

  Then we can write the contribution of cell $K$ to the right hand side
  component $F_i$ as
  \begin{eqnarray*}
    F_i|_K
    &=&
    \left\{ \int_K f(x) \varphi_g^i(x) \varphi_{g'}^j(x) 
    \ dx \right\} \phi_{g'}^j 
    \\
    &=&
    \left\{
    \sum_{0\le c<2^{\texttt{dim}}}
    B_c^{il} \int_{K_c} f(x) \varphi_{g'}^l(x) \varphi_{g'}^j(x) 
    \ dx \right\} \phi_{g'}^j.
  \end{eqnarray*}
  In matrix notation, this can be written as
  \begin{eqnarray*}
    F_i|_K
    =
    \sum_{0\le c<2^{\texttt{dim}}}
    F_i|_{K_c},
    \qquad
    \qquad
    F_i|_{K_c} = B_c^{il} M_{K_c}^{lj}  \phi_{g'}^j
    = (B_c M_{K_c})^{il} \phi_{g'}^j,
  \end{eqnarray*}
  where $M_{K_c}^{lj}=\int_{K_c} f(x) \varphi_{g'}^l(x) \varphi_{g'}^j(x)$ is
  the weighted mass matrix on child $c$ of cell $K$.
  
  The next question is what happens if a child $K_c$ of $K$ is not
  active. Then, we have to apply the process recursively, i.e. we have to
  interpolate the basis functions $\varphi_g^i$ onto child $K_c$ of $K$, then
  onto child $K_{cc'}$ of that cell, onto child $K_{cc'c''}$ of that one, etc,
  until we find an active cell. We then have to sum up all the contributions
  from all the children, grandchildren, etc, of cell $K$, with contributions
  of the form
  \begin{equation}
    F_i|_{K_{cc'}} = (B_cB_{c'} M_{K_{cc'}})^{ij}  \phi_{g'}^j,
  \end{equation}
  or
  \begin{equation}
    F_i|_{K_{cc'c''}} = (B_c B_{c'} B_{c''}M_{K_{cc'c''}})^{ij}
    \phi_{g'}^j,
  \end{equation}
  etc. We do this process recursively, i.e. if we sit on cell $K$ and see that
  it has children on grid $g'$, then we call a function
  \texttt{assemble\_case\_2} with an identity matrix; the function will
  multiply it's argument from the left with the prolongation matrix; if the
  cell has further children, it will call itself with this new matrix,
  otherwise it will perform the integration.

\item[(iii)] The last case is where $K$ is active on mesh $g'$ but not mesh
  $g$. In that case, we have to express basis function $\varphi_{g'}^j$ in
  terms of the basis functions defined on the children of cell $K$, rather
  than $\varphi_g^i$ as before. This of course works in exactly the same
  way. If the children of $K$ are active on mesh $g$, then
  leading to the expression
  \begin{eqnarray*}
    F_i|_K
    &=&
    \left\{ \int_K f(x) \varphi_g^i(x) \varphi_{g'}^j(x) 
    \ dx \right\} \phi_{g'}^j 
    \\
    &=&
    \left\{
    \sum_{0\le c<2^{\texttt{dim}}}
    \int_{K_c} f(x) \varphi_{g'}^i(x) B_c^{jl} \varphi_{g'}^l(x) 
    \ dx \right\} \phi_{g'}^j.
  \end{eqnarray*}
  In matrix notation, this expression now reads as
  \begin{eqnarray*}
    F_i|_K
    =
    \sum_{0\le c<2^{\texttt{dim}}}
    F_i|_{K_c},
    \qquad
    \qquad
    F_i|_{K_c} = M_{K_c}^{il} B_c^{jl}  \phi_{g'}^j
    =
    (M_{K_c} B_c^T)^{ij} \phi_{g'}^j,
  \end{eqnarray*}
  and correspondingly for cases where cell $K$ is refined more than once on
  mesh $g$:
  \begin{equation}
    F_i|_{K_{cc'}} = (M_{K_{cc'}} B_{c'}^T B_c^T)^{ij}  \phi_{g'}^j,
  \end{equation}
  or
  \begin{equation}
    F_i|_{K_{cc'c''}} = (M_{K_{cc'c''}} B_{c''}^T B_{c'}^T B_c^T)^{ij}
    \phi_{g'}^j,
  \end{equation}
  etc. In other words, the process works in exactly the same way as before,
  except that we have to take the transpose of the prolongation matrices and
  need to multiply it to the mass matrix from the other side. 
\end{itemize}


The expressions for cases (ii) and (iii) can be understood as repeatedly
interpolating either the left or right basis functions in the scalar product
$(f \varphi_g^i, \varphi_{g'}^j)_K$ onto child cells, and then finally
forming the inner product (the mass matrix) on the final cell. To make the
symmetry in these cases more obvious, we can write them like this: for case
(ii), we have
\begin{equation}
  F_i|_{K_{cc'\cdots c^{(k)}}}
  = [B_c B_{c'} \cdots B_{c^{(k)}} M_{K_{cc'\cdots c^{(k)}}}]^{ij}
    \phi_{g'}^j,
\end{equation}
whereas for case (iii) we get
\begin{equation}
  F_i|_{K_{cc'\cdots c^{(k)}}}
  = [(B_c B_{c'} \cdots B_{c^{(k)}} M_{K_{cc'\cdots c^{(k)}}})^T]^{ij}
    \phi_{g'}^j,
\end{equation}



\subsection{Description of the test case}

A nuclear reactor core is composed of different types of assemblies. An
assembly is essentially the smallest unit that can be moved in and out of a
reactor, and is usually rectangular or square. However, assemblies are not
fixed units, as they are assembled from a complex lattice of different fuel
rods, control rods, and instrumentation elements that are held in place
relative to each other by spacers that are permanently attached to the rods.
To make things more complicated, there are different kinds of assemblies that
are used at the same time in a reactor, where assemblies differ in the type
and arrangement of rods they are made up of.

Obviously, the arrangement of assemblies as well as the arrangement of rods
inside them affect the distribution of neutron fluxes in the reactor (a fact
that will be obvious by looking at the solution shown below in the results
sections of this program). Fuel rods, for example, differ from each other in
the enrichment of U-235 or Pu-239. Control rods, on the other hand, have zero
fission, but nonzero scattering and absorption cross sections.

This whole arrangement would make the description or spatially dependent
material parameters very complicated. It will not become much simpler, but we
will make one approximation: we merge the volume inhabited by each cylindrical
rod and the surrounding water into volumes of quadratic cross section into
so-called ``pin cells'' for which homogenized material data are obtained with
nuclear database and knowledge of neutron spectrum. The homogenization makes
all material data piecewise constant on the solution domain for a reactor with
fresh fuel. Spatially dependent material parameters are then looked up for the
quadratic assembly in which a point is located, and then for the quadratic pin
cell within this assembly.

In this tutorial program, we simulate a quarter of a reactor consisting of $4
\times 4$ assemblies. We use symmetry (Neumann) boundary conditions to reduce
the problem to one quarter of the domain, and consequently only simulate a
$2\times 2$ set of assemblies. Two of them will be UO${}_2$ fuel, the other
two of them MOX fuel. Each of these assemblies consists of $17\times 17$ rods
of different compositions. In total, we therefore create a $34\times 34$
lattice of rods. To make things simpler later on, we reflect this fact by
creating a coarse mesh of $34\times 34$ cells (even though the domain is a
square, for which we would usually use a single cell). In deal.II, each cell
has a \texttt{material\_id} which one may use to associated each cell with a
particular number identifying the material from which this cell's volume is
made of; we will use this material ID to identify which of the 8 different
kinds of rods that are used in this testcase make up a particular cell. Note
that upon mesh refinement, the children of a cell inherit the material ID,
making it simple to track the material even after mesh refinement.

The arrangement of the rods will be clearly visible in the images shown in
the results section. The cross sections for materials and for both energy
groups are taken from a OECD/NEA benchmark problem. The detailed configuration
and material data is given in the code.


\subsection{What the program does (and how it does that)}

As a coarse overview of what exactly the program does, here is the basic
layout: starting on a coarse mesh that is the same for each energy group, we
compute inverse eigenvalue iterations to compute the $k$-eigenvalue on a given
set of meshes. We stop these iterations when the change in the eigenvalue
drops below a certain tolerance, and then write out the meshes and solutions
for each energy group for inspection by a graphics program. Because the meshes
for the solutions are different, we have to generate a separate output file
for each energy group, rather than being able to add all energy group
solutions into the same file.

After this, we evaluate the error indicators as explained in one of the sections
above for each of the meshes, and refine and coarsen the cells of each mesh
independently. Since the eigenvalue iterations are fairly expensive, we don't
want to start all over on the new mesh; rather, we use the SolutionTransfer
class to interpolate the solution on the previous mesh to the next one upon
mesh refinement. A simple experiment will convince you that this is a lot
cheaper than if we omitted this step. After doing so, we resume our eigenvalue
iterations on the next set of meshes.

The program is controlled by a parameter file, using the ParameterHandler
class already mentioned in the step-19 example program. We will show a
parameter file in the results section of this section. For the moment suffice
it to say that it controls the polynomial degree of the finite elements used,
the number of energy groups (even though all that is presently implemented are
the coefficients for a 2-group problem), the tolerance where to stop the
inverse eigenvalue iteration, and the number of refinement cycles we will do.

\end{document}
