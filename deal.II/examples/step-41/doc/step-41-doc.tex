\documentclass{article}

\usepackage{amsmath}
\usepackage{amssymb}

\title{Documentation of step-41, The obstacle problem}
\author{Joerg Frohne}
\date{November 11th, 2011}
\begin{document}
\maketitle

\section{Introduction}

This example is based on the Laplace equation in a two-dimensional space $\Omega = \left[-1,1\right]^2$. It shows how to handle an obstacle problem. Therefore we have to solve a variational inequality. We will derive it from classical formulation.\\
As a physical interpretation you imagine a membrane which is fixed on the boundary $\partial\Omega$. The membrane shows elastic material behavior with Young's modulus $E = 1.0$ for simplicity and there acts a force like from the earth gravitation on it. So the membrane dents in one direction and hits the cascaded obstacle which is described by the function $g$.

\section{Classical formulation}

The classical formulation of the problem possesses the following form:
\begin{align}
 -div (\sigma) &\geq f & &\quad\text{in } \Omega,\\
 \sigma &= E\nabla u & &\quad\text{in } \Omega,\\
 u(x,y) &= 0 & &\quad\text{on }\partial\Omega,\\
(\Delta u + f)(u - g) &= 0 & &\quad\text{in } \Omega,\\
 u(x,y) &\geq g(x,y) & &\quad\text{in } \Omega
\end{align}
with $u\in H^2(\Omega)$.

\noindent
$u$ is a scalar valued function that denotes the displacement of the membrane. The first equation is called equilibrium condition with the force of areal density $f$. The second one is known as Hooke's Law with the stresses $\sigma$. At the boundary we have zero Dirichlet conditions. And (4) together with the last inequality builds the obstacle condition which has to hold for the hole domain.\\
In this case it is possible to join the first two equations which yields the Laplace equation:
\begin{equation}
 -\Delta u(x,y) \geq f(x,y)\quad\text{in }(x,y)\in \Omega.
\end{equation}
As mentioned above we choose $E=1.0$ for simplicity.

\section{Derivation of the variational inequality}

An obvious way to obtain the variational formulation of the obstacle problem is to consider the total potential energy:
\begin{equation}
 E(u):=\dfrac{1}{2}\int\limits_{\Omega} \nabla u \cdot \nabla - \int\limits_{\Omega} fu.
\end{equation}
We have to find a solution $u\in G$ of the following minimization problem
\begin{equation}
 E(u)\leq E(v)\quad \forall v\in G,
\end{equation}
with the convex set of admissble displacements:
\begin{equation}
 G:=\lbrace v\in V: v\geq g \text{ a.e. in } \Omega\rbrace,\quad V:=H^1_0(\Omega).
\end{equation}
This set takes care of the conditions (3) and (5).\\
Now we consider a function
\begin{equation}
 F(\varepsilon) := E(u+\varepsilon(v-u)),\quad\varepsilon\in\left[0,1\right],\quad u,v\in G,
\end{equation}
which takes its minimum at $\varepsilon = 0$, so that $F'(0)\geq 0$. Note that $u+\varepsilon(v-u) = (1-\varepsilon)u+\varepsilon v\in G$ because of the convexity of $G$. If we compute $F'(\varepsilon)\vert_{\varepsilon=0}$ it yields the variational formulation we are searching for:\\
\textit{Find a function $u\in G$ with}
\begin{equation}
 \left(\nabla u, \nabla(v-u)\right) \geq \left(f,v-u\right) \quad \forall v\in G.
\end{equation}
For the equivalent saddle point formulation of this problem we introduce a Lagrange multiplier $\lambda$ and the convex cone $K\subset W:=V^*$ of Lagrange multipliers. This yields to:\\
Find $u\in V$ and $\lambda\in K$ such that
\begin{eqnarray}
 a(u,v) + b(v,\lambda) &=& f(v),\quad v\in V\\
 b(u,\mu - \lambda) &\leq& \langle g,(\mu - \lambda)\rangle,\quad\mu\in K,
\end{eqnarray}
with
\begin{eqnarray}
 a(u,v) &:=& \left(\nabla u, \nabla v\right),\quad u,v\in V\\
 b(u,\mu) &:=& (u,\mu),\quad u\in V,\quad\mu\in W.
\end{eqnarray}
The existence and uniqueness of $(u,\mu)\in V\times K$ of the saddle point problem (14) and (15) has been stated in Grossmann and Roos: Numerical treatment of partial differential equations, Springer-Verlag, Heidelberg-Berlin, 2007, 596 pages, ISBN 978-3-540-71582-5.



\section{Active Set methods to solve (11)}

There are different methods to solve the variational inequality. As one possibility you can understand (11) as a convex quadratic program (QP) with inequality constraints.\\
After we discretized the saddle point problem, we obtain the following system of equations and inequalities for $p\in\mathcal{S}:=\Omega_h\backslash\partial\Omega_h$:
\begin{eqnarray}
 &A_h u_h + B_h\lambda_h = f_h,&\\
 &u_{n,p} \leq g_p,\quad \lambda_p \geq 0,\quad \lambda_p(u_{n,p} - g_p) = 0.&
\end{eqnarray}
with $u_{n,p}:=D_{pp} u_p\leq g_p, p\in S$ as a non-pentration condition. The matrix $B_h$ has the form $B_h:=D$ where $D$ is a diagonal matrix with the entries
\begin{equation}
 D_{pp} := \int\limits_{\Omega}\varphi_p^2 dx,\quad p\in\mathcal{S}.
\end{equation}
Now we define for each vertex $p\in \mathcal{S}$ the function
\begin{equation}
 C(u_{n,p},\lambda_p):=\lambda_p - \max\lbrace 0, \lambda_p + c( u_{n,p} - g_p\rbrace,\quad c>0.
\end{equation}
So we can express the conditions in (17) as
\begin{equation}
 C(u_{n,p},\lambda_p) = 0,\quad p\in\mathcal{S}.
\end{equation}
The primal-dual active set strategy is an iterative scheme which is based on (19) to predict the next active and inactive sets $\mathcal{A}_k$ and $\mathcal{F}_k$. (See Hintermueller, Ito, Kunisch: The primal-dual active set strategy as a semismooth newton method, SIAM J. OPTIM., 2003, Vol. 13, No. 3, pp. 865-888.)\\
% \begin{eqnarray}
%  \min\limits_{u_h} q(u_h) &=& \dfrac{1}{2}u_h^TAu_h + u_h^Tb\\
%  \text{subject to}\quad c_i^Tu_h &=& 0,\quad i\in I_{\partial\Omega}\\
%  c_i^T u_h &\geq& g_i,\quad i\in I_{\Omega}.
% \end{eqnarray}
% In this formulation $A$ is the mass matrix with $A_{ij} = \left(\nabla\varphi_i,\nabla\varphi_j\right)$ which includes the Dirichlet-Boundary conditions and $b$ is the right-hand-side with $b_i = \left(f_i,\varphi_i\right)$. $u_h$ and $c$ are also vectors with the same dimension as $b$.\\
The algorithm for primal-dual active set method works as follows:
\begin{itemize}
 \item [(0)] Initialize $\mathcal{A}_k$ and $\mathcal{F}_k$, such that $\mathcal{S}=\mathcal{A}_k\cup\mathcal{F}_k$ and $\mathcal{A}_k\cap\mathcal{F}_k=\O{}$ and set $k=1$.
 \item [(1)] Find the primal-dual pair $(u^k_h,\lambda^k_h)$\\
 \begin{equation}
 \begin{split}
  A_h u^k_h + B_h\lambda^k_h = f_h,\\
  u^k_{n,p} = g_p\quad\forall p\in\mathcal{A}_k,\\
  \lambda_p = 0\quad\forall p\in\mathcal{F}_k.
 \end{split}
 \end{equation}
 \item [(2)] Define the new active and inactive sets by
 \begin{equation}
 \begin{split}
  \mathcal{A}_{k+1}:=\lbrace p\in\mathcal{S}:\lambda^k_p + c(u^k_{n,p} - g_p)> 0\rbrace,\\
  \mathcal{F}_{k+1}:=\lbrace p\in\mathcal{S}:\lambda^k_p + c(u^k_{n,p} - g_p)\leq 0\rbrace.
 \end{split}
 \end{equation}
 \item [(3)] If $\mathcal{A}_{k+1}=\mathcal{A}_k$ and $\mathcal{F}_{k+1}=\mathcal{F}_k$ then stop, else set $k=k+1$ and go to step (1).
%  \item [(2)] Solve $A^k u_h^k = b^k$.
%  \item [(3)] Error control
%  \item [(4)] Compute $res =  b - Au_h^k$
%  \item [(5)] Set $u_h^{k+1} = u_h^k,\quad k = k+1$ and go to step (1).
\end{itemize}
For any the primal-dual pair $(u^k_h,\lambda^k_h)$ that satisfies the conditions in step (3), we differ between three cases:
\begin{itemize}
 \item [1.] $\lambda^k_p + c(u^k_{n,p} - g_p)> 0$ (p active):\\
  Then either $u^k_{n,p}>g_p$ and $\lambda^k_{n,p}=0$ (pentration) or $\lambda^k_{n,p}>0$ and $u^k_{n,p}=g_p$ (pressing load).
 \item [2.] $\lambda^k_p + c(u^k_{n,p} - g_p)\leq 0$ (p inactive):\\
  Then either $u^k_{n,p}\leq g_p$ and $\lambda^k_{n,p}=0$ (no contact) or $\lambda^k_{n,p}\leq0$ and $u^k_{n,p}=g_p$ (unpressing load).
\end{itemize}
Now we want to show the slantly derivation function of $C(.,.)$:
\begin{equation}
 \dfrac{\partial}{\partial u^k_p}C(u^k_p,\lambda^k_p) = \begin{cases}
                                   -cD_{pp},\quad \lambda^k_p + c(u^k_{n,p} - g_p)> 0\\
                                   0\lambda^k_p,\quad \lambda^k_p + c(u^k_{n,p} - g_p)\leq 0.
                                  \end{cases}
\end{equation}
\begin{equation}
 \dfrac{\partial}{\partial\lambda^k_p}C(u^k_p,\lambda^k_p) = \begin{cases}
                                   0,\quad \lambda^k_p + c(u^k_{n,p} - g_p)> 0\\
                                   \lambda^k_p,\quad \lambda^k_p + c(u^k_{n,p} - g_p)\leq 0.
                                  \end{cases}
\end{equation}
This suggest a semismooth Newton step of the form
\begin{equation}
 \begin{pmatrix}
 A_{\mathcal{F}_k\mathcal{F}_k} & A_{\mathcal{F}_k\mathcal{A}_k} & D_{\mathcal{F}_k} & 0\\
 A_{\mathcal{A}_k\mathcal{F}_k} & A_{\mathcal{A}_k\mathcal{A}_k} & 0 & D_{\mathcal{A}_k}\\
 0 & 0 & Id_{\mathcal{F}_k} & 0\\
 0 & -cD_{\mathcal{A}_k} & 0 & 0
\end{pmatrix}
\begin{pmatrix}
 \delta u^k_{\mathcal{F}_k}\\ \delta u^k_{\mathcal{A}_k}\\ \delta \lambda^k_{\mathcal{F}_k}\\ \delta \lambda^k_{\mathcal{A}_k}
\end{pmatrix}
=
-\begin{pmatrix}
 (Au^k + \lambda^k - f)_{\mathcal{F}_k}\\ (Au^k + \lambda^k - f)_{\mathcal{A}_k}\\ \lambda^k_{\mathcal{F}_k}\\ -c(D_{\mathcal{A}_k} u^k - g)_{\mathcal{A}_k}
\end{pmatrix}.
\end{equation}
The algebraic representation of (20) follows now by setting $\delta u^k := u^{k+1} - u^k$ and $\delta \lambda^k := \lambda^{k+1} - \lambda^k$
\begin{equation}
\begin{pmatrix}
 A_{\mathcal{F}_k\mathcal{F}_k} & A_{\mathcal{F}_k\mathcal{A}_k} & D_{\mathcal{F}_k} & 0\\
 A_{\mathcal{A}_k\mathcal{F}_k} & A_{\mathcal{A}_k\mathcal{A}_k} & 0 & D_{\mathcal{A}_k}\\
 0 & 0 & Id_{\mathcal{F}_k} & 0\\
 0 & D_{\mathcal{A}_k} & 0 & 0
\end{pmatrix}
\begin{pmatrix}
 u^k_{\mathcal{F}_k}\\ u^k_{\mathcal{A}_k}\\ \lambda^k_{\mathcal{F}_k}\\ \lambda^k_{\mathcal{A}_k}
\end{pmatrix}
=
\begin{pmatrix}
 f_{\mathcal{F}_k}\\ f_{\mathcal{A}_k}\\ 0\\ g_{\mathcal{A}_k}
\end{pmatrix}.
\end{equation}
It's easy to see that we can eliminate the third row and the third column because it implies $\lambda_{\mathcal{F}_k} = 0$:
\begin{equation}
\begin{pmatrix}
 A_{\mathcal{F}_k\mathcal{F}_k} & A_{\mathcal{F}_k\mathcal{A}_k} & 0\\
 A_{\mathcal{A}_k\mathcal{F}_k} & A_{\mathcal{A}_k\mathcal{A}_k} & D_{\mathcal{A}_k}\\
 0 & D_{\mathcal{A}_k} & 0
\end{pmatrix}
\begin{pmatrix}
 u^k_{\mathcal{F}_k}\\ u^k_{\mathcal{A}_k}\\ \lambda^k_{\mathcal{A}_k}
\end{pmatrix}
=
\begin{pmatrix}
 f_{\mathcal{F}_k}\\ f_{\mathcal{A}_k}\\ g_{\mathcal{A}_k}
\end{pmatrix}.
\end{equation}
And it yields
\begin{equation}
 \lambda_h = D^{-1}\left(f_{\mathcal{S}} - A_{\mathcal{S}}u_{\mathcal{S}}\right).
\end{equation}


Finally we have to solve linear problems for what we use CG-Solver with a AMG preconditioner from Trilinos.

\end{document}