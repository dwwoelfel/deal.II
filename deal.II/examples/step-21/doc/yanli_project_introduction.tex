\documentclass[12pt]{article}
\topmargin 0pt \oddsidemargin 0pt \evensidemargin 0pt
\textwidth=14truecm \textheight=21.5truecm
\renewcommand{\baselinestretch}{1.5}
\begin{document}
\author{Yan Li}
\title{A Numerical Simulation for Two Phase Flow}
\maketitle
\section{Introduction}
\subsection{Abstract}

In this project, we propose a numerical simulation for two phase
flow problem in porous media. The two phase flow system includes one
elliptic equation and one nonlinear transport equation. We apply
mixed finite element method and Discontinuous Galerkin method for
this system. Some numerical results for two dimensional case are
given by $RT_{0}\times DQ_{0}\times DQ_{0}$.
\\
The numerical computation is based on $dealII$. We use vector shape
functions from step9, DG method from step12, mixed method and Schur
complement from step20 and many many useful tools from the library.

\subsection{Two Phase Flow Problem }
The modeling of two phase flow in porous media is important for both
environmental rededication and the management of petroleum
reservoirs. Practical situations involving two phase flow include
the dispersal of a nonaqueous phase liquid in an aquifer or the
displacement of a non-aqueous heterogeneity on the flow and
transport. Simulation models, if they are to provide realistic
predictions, must accurately account for these effects.
%However,
%because permeability heterogeneity occurs at many different length
%scales, numerical flow models cannot in general resolve all of the
%scales of variation.Therefore, approaches are needed for
%representing the effects of subgrid scale variations on larger scale
%flow results. Typically, upscaled or multiscale models are employed
%for such systems. \\
In our project,we consider a kind of periodic permeability,our
numerical result shows that the heterogeneity effects are simulated accurately.\\
Consider two phase flow in a reservoir $\Omega$ under the assumption
that the displacement is dominated by viscous effects; i.e. we
neglect the effects of gravity, compressibility, and capillary
pressure. Porosity will be considered to be constant. The two phase
will be referred to as water and oil, designated by subscripts $w$
and $o$, respectively. We write Darcy's for each phase as follows:
\begin{eqnarray}
\mathbf{u}_{j} = \frac{k_{rj}(S)}{\mu_{j}} \mathbf{K} \cdot \nabla p
\end{eqnarray}
\indent where, $\mathbf{u}_{j}$ is the phase velocity, $K$ is the
permeability tensor, $k_{rj}$ is the relative permeability to phase
$j$($j=o,w$),$S$ is the water saturation(volume fraction), $P$ is
pressure and $\mu_{j}$ is the viscosity of phase $j$($j=o,w$).\\
Combining Darcy's law with a statement of conservation of mass
allows us to express the governing equations in terms of the
so-called pressure and saturation equations:
\begin{eqnarray}
\nabla \cdot (\mathbf{K}(x,y)\lambda(S) \nabla p)= q(x,y) && \forall(x,y)\in\Omega\\
 S_{t} + \mathbf{u} \cdot \nabla F(S) = 0&& \forall(x,y)\in\Omega
\end {eqnarray}


\indent where, $\lambda$ is the total mobility, f is the fractional
flow of water, $q$ is a source term and $\mathbf{u}$ is the total
velocity, which are respectively given by:
$$\mathbf{u} =
\mathbf{u}_{o} + \mathbf{u}_{w} = -\lambda(S) \mathbf{K}\cdot\nabla
p$$
$$\lambda(S) = \frac{k_{rw}(S)}{\mu_{w}}+\frac{k_{ro}(S)}{\mu_{o}}$$
$$F(S) = \frac{k_{rw}(S)/\mu_{w}}{k_{rw}(S)/\mu_{w} + k_{ro}(S)/\mu_{o}}$$


\subsection{Discretization}

 For simplicity, in our project we will assume no
source $q=0$ and the heterogeneous porous medium is isotropic
$\mathbf{K}(x,y) =
k(x,y) \mathbf{I}$. \\
Our two dimensional numerical simulation will be done on unit cell
$\Omega = [0,1]\times [0,1]$ for $t\in [0,T]$.
\begin {eqnarray}
\mathbf{u}(x,y)+\mathbf{K}(x,y)\lambda(S) \nabla p= 0 && \forall(x,y)\in\Omega, \forall t\in [0,T]\\
\nabla \cdot\mathbf{u}(x,y)= 0 && \forall(x,y)\in\Omega, \forall t \in [0,T] \\
S_{t} + \mathbf{u} \cdot \nabla F(S) = 0&& \forall(x,y)\in\Omega,
\forall t \in [0,T]
\end {eqnarray}
 Boundary conditions are:
\[
\begin {array}{cr}
p(x,y)=1 & \forall(x,y)\in \Gamma_{1}:=\{(x,y)\in \partial \Omega: x=0\}\\
p(x,y)=0 & \forall(x,y)\in  \Gamma_{2}:=\{(x,y)\in \partial \Omega: x=1\}\\
\mathbf{u}(x,y)\cdot \mathbf{n}=0 & \forall(x,y)\in
\partial\Omega \setminus(\Gamma_{1}\bigcup \Gamma_{2})
\end {array}
\]
\\
Initial conditions are:
\[
\begin {array}{cr}
S(x,y,t=0)= 1& \forall (x,y) \in \Gamma_{1}\\
S(x,y,t=0) = 0 & \forall(x,y)\in \partial \Omega \setminus
\Gamma_{1}
\end {array}
\]
\\
We apply mixed finite method on velocity and pressure. To be
well-posed, we choose Raviart-Thomas spaces $RT_{k}$ for
$\mathbf{u}$ and discontinuous elements of class $DQ_{k}$ for $p$,
then the mixed
system is:\\
Find $(\mathbf{u},p)\in RT_{k}\times DQ_{k}$ such that:
\begin {eqnarray}
\sum_{\kappa}\{ \int _{\kappa}(K \lambda)^{-1} \mathbf{u}\cdot
\mathbf{v} dx - \int_{\kappa} p \nabla \cdot \mathbf{v} dx\}
 =- \int_{\Gamma _{1}} \mathbf{v}\cdot \mathbf{n}&&  \forall\mathbf{v}\in RT_{k}(\Omega)\\
\sum_{\kappa}\{\int (\nabla \cdot \mathbf{u}) q dx\} = 0 && \forall
q\in DQ_{k}(\Omega)
\end {eqnarray}
For saturation, we also use discontinuous finite element method.
i.e. Find $S^{n+1} \in DQ_{k}$ such that for all $ \phi \in DQ_{k}$,
the following formulation holds:
\begin {eqnarray}
\sum_{\kappa}\{\int_{\kappa}\frac{S^{n+1}-S^{n}}{\triangle t} \phi
dx + \int_{\kappa} (\mathbf{u}^{n+1}\cdot \nabla F(S^{n})) \phi
dx\} =0
\end {eqnarray}
Integrating by parts:
\begin {eqnarray}
\nonumber
 \sum_{\kappa}\{\int_{\kappa}S^{n+1} \phi dx +\triangle t
\int_{\partial \kappa}F(S^{n})( \mathbf{u}^{n+1}\cdot \mathbf{n} )
\phi dx &-\triangle t\int_{\kappa}  F(S^{n})( \mathbf{u^{n+1}}\cdot
\nabla
\phi )dx\}\\
&= \sum_{\kappa}\int_{\kappa} S^{n} \phi dx
\end {eqnarray}

\indent where,$\mathbf{n}$ denotes the unit outward normal to the
boundary $\partial \kappa$. And here we can use $u^{n+1}$ instead of
$u^{n}$ is because that we view $(u^{n+1},p^{n+1},S^{n+1})$ as
a block vector,$u^{n+1}$ could be implement in the coefficient function for saturation.
We believe the saturation is computed more accurately in this way.\\
Considering the discontinuity of the discrete function $S_h$ on
interelement faces, the flux $\mathbf{u}^{n+1}\cdot \mathbf{n} $ is
computed as:
 \begin{eqnarray}
&&\int_{\partial \kappa}F(S^{n}) (\mathbf{u}^{n+1}\cdot \mathbf{n})
\phi dx =\\
\nonumber && \int_{\partial \kappa _{+}}
F(S^{n,+})(\mathbf{u}^{n+1,+}\cdot \mathbf{n})\phi dx
+\int_{\partial \kappa _{-}} F(S^{n,-})(\mathbf{u}^{n+1,-}\cdot
\mathbf{n})\phi dx
\end{eqnarray}

where, $\partial \kappa _{-}:= \{x\in
\partial\kappa , \mathbf{u}(x) \cdot \mathbf{n}<0\}$ denotes the inflow boundary
and$\partial \kappa _{+}:= \{\partial \kappa \setminus \partial
\kappa_{-}\}$ is the outflow part of the boundary. By the
discontinuity of$ S_{h}$ , $F(S^{n,-})$ takes the value of
neighboring cell,$F(S^{n+})$ takes the value of cell $\kappa$.

\subsection{Implementation}
We use
$dealII$ to implement our mixed and DG system. The main idea is same
with step-20 but there are some new problems we have to consider:\\
\indent $(1)$ We has the three blocks vector $(u,p,S)$ , in which
all the functions are dependent on time. i.e. At each time step we
need project the $solution$ into $old-solution$, using
$old-solution$ to get a new $solution$.
Keep doing this until the last time step;\\
At time $t=t^{n+1}$ , suppose $old-solution=(u^{n},p^{n},S^{n})$ is
known, in $assemble-system()$ part, we assemble system matrix as:
\[
\begin {array}{cccccccccccc}
\lceil &M(S^{n}) &B^{T}& 0 &\rceil & \lceil& \mathbf{u}^{n+1}&\rceil& &\lceil& 0 &\rceil\\
|      &B&    0 & 0 & |     &|      & p^{n+1} &|        &=&|     & q &|\\
\lfloor&\triangle t \nabla F(S^n)&    0& I & \rfloor & \lfloor
&S^{n+1} & \rfloor & & \lfloor& S^{n}& \rfloor
\end {array}
\]
\\
In $solve()$ part, we solve the first two equations independent of
the third equation, since $M( S^n)$ is already known. As in step-20,
using vector base functions, Schur complement with a
preconditioner and CG method, we get $u^{n+1}$and $p^{n+1}$. \\
Then, with the above $u^{n+1}$ and $p^{n+1}$, we could compute
$S^{n+1}$ by :
\begin {eqnarray}
\sum_{\kappa}\int_{\kappa}S^{n+1} \phi dx&&=
\sum_{\kappa}\{\int_{\kappa} S^{n} \phi dx+\Delta t\int_{\kappa}
F(S^{n}) \mathbf{u^{n+1}}\cdot \nabla \phi dx\\
\nonumber && -\Delta t \int_{\partial \kappa_{-}}F(S^{n,-})
\mathbf{u}^{n+1,-}\cdot \mathbf{n} \phi dx -\Delta t \int_{\partial
\kappa_{+}}F(S^{n}) \mathbf{u}^{n+1}\cdot \mathbf{n} \phi dx\}
\end {eqnarray}
Now, project solution $(u^{n+1},p^{n+1},S^{n+1})$ into
$old-solution$, do the above process for next time step.\\
 \indent
$(2)$ The numerical flux term is related with neighbor cells.In our
implementation $solve( )$, we do the following on each cell: \\
For each face, compute the flux $\mathbf{u}\cdot F(S)$, the flux is
negative means it is an in-flow face. Then if this in-flow face is
on the boundary $\Gamma_{1}$:$F(S^{-})=F(1)$; If the in-flow
face is not on boundary, $F(S^{-})=F(S|_{neighbor})$.
Flux is positive means it is an out-flow face, we just use $ F(S)$ on current cell.\\
All the other functions are commented in code, please see next part
- the commented program.

\subsection{Test Case}
Our two phase flow are chosen as water and oil. The total mobility
is : $$\lambda (S) = \frac{1.0}{\mu} S^2 +(1-S)^2$$ Permeability is
:
$$K(x,y)=\mathbf{k}(x,y)I=\frac{1.0}{2+1.99\sin(2\pi\frac{2x-y}{\epsilon})}
I$$
 Fractional flow of water is: $$F(S)=\frac{S^2}{S^2+\mu (1-S)^2}$$
Choose $\epsilon=0.05$ , viscosity $\mu=0.2$. \\
The resulting solution will be shown in result part.

\end{document}
