\documentclass{article}
\usepackage{amsmath}

\begin{document}

\section{Foreword}

This program demonstrates a number of techniques that have not been shown in
previous example programs. In particular, it shows how to program for
one-dimensional problems, and some aspects of what to do with nonlinear
problems, in particular how to transfer the solution from one grid to the next
finer one. Apart from this, however, the program does not attempt to do much
more than to entertain those who sometimes like to play with maths.

The application we chose is, as you will see, not even very well suited for
anything, since it is rather impossible to solve. When I started to write the
program, I was not aware of this, and it only turned out later that the
optimization problem we are looking at here is severely plagued by many,
likely even degenerate minima, and that we cannot really hope to find a global
one. What we do instead is to rather start the optimization from many initial
guesses (which is cheap since the problem is 1d), and hope that we can get a
reasonable best solution for some of them. While the whole thing, as an
application, is not very satisfactory, keep in mind that solving particular
applications is not the goal of the tutorial programs; rather, we would like
to demonstrate techniques of programming with deal.II, which is indeed the
focus here.


\section{The problem}

Now for a description of the problem.  In the book by Dacorogna on the
Calculus of Variations, I found the following statement, which confused me
tremendously at first (see Section 3.4.3, ``Lavrentiev Phenomenon'', very
slightly edited):

\begin{quote}
  \textbf{Theorem 4.6:} Let 
  $$I(u)=\int_0^1 (x-u^3)^2 (u')^6\; dx.$$
  Let
  $$
    {\cal W}_1 = \{ u\in W^{1,\infty}(0,1) : u(0)=0, u(1)=1 \}
    $$ $$
    {\cal W}_2 = \{ u\in W^{1,1}(0,1) : u(0)=0, u(1)=1 \}
  $$

  Then
  $$
  \inf_{u\in {\cal W}_1} I(u) \ge c_0 > 0 = \inf_{u\in {\cal W}_2} I(u).
  $$
  Moreover the minimum of $I(u)$ over ${\cal W}_2$ is attained by
  $u(x)=x^{1/3}$. 

  \textsc{Remarks.}

  [\ldots]

  ii) it is interesting to note that if one uses the usual finite element
  methods (by taking piecewise affine functions, which are in $W^{1,\infty}$)
  one will not be able to detect the minimum of some integrals such as the one
  in the theorem.
\end{quote}
In other words: minimizing the energy functional over one space
($W^{1,\infty}$) does not give the same value as when minimizing over a larger
space ($W^{1,1}$). Furthermore, they give a rough estimate of the value of the
constant $c_0$, which is $c_0=\tfrac{7^23^5}{2^{18}5^5}\approx 1.61\cdot
10^{-6}$ (although by their calculation it is obvious that this estimate is
far too small, but the point of course is just to show that it is strictly
larger than zero).

While the theorem was not surprising, the remark stunned me at first. After
all, we know that we can approximate functions in $W^{1,1}$ to arbitrary
accuracy.  Also, although it is true that finite element functions are in
$W^{1,\infty}$, this statement is not really accurate: if the function itself
is bounded pointwise by, say, a constant $C$, then its gradient is bounded by
$2C/h$, and thus $\|u_h\|_{1,\infty} \le 2C/h$. That means that we should be
able to lift this limit just by mesh refinement. Finite element functions are
therefore only in $W^{1,\infty}$ if one considers them on a fixed grid, not on
a sequence of successively finer grids. (Note, we can only lift the
boundedness in $W^{1,1}$ in the same way by considering functions that
oscillate at cell frequency; these, however, do not converge in any reasonable
measure.)

So it took me a while to see where the problem lies. Here it is: While we are
able to approximate functions to arbitrary accuracies in \textit{Sobolev
  norms}, this does not necessarily also hold with respect to the functional
$I(u)$!  After all, this functional was made to show exactly these
pathologies.

What happens in this case is actually not so difficult to understand. Let us
look at what happens if we plug the lowest-order (piecewise linear)
interpolant $i_hu$ of the optimal solution $u=x^{1/3}$ into the functional
$I(u)$: on the leftmost cell, the left end of $i_hu$ is tagged to zero by the
boundary condition, and the right end has the value $i_hu(h)=u(h)=h^{1/3}$. So
let us only consider the contribution of this single cell to $I(u)$:
\begin{eqnarray*}
  \int_0^h (x-(i_hu)^3)^2 ((i_hu)')^6 dx
  &=&
  \int_0^h (x-(h^{1/3}x)^3)^2 ((h^{1/3}/h)')^6 dx
  \\
  &=&
  h^{-4} \int_0^h (x^2-2hx^4+h^2x^6) dx
  \\
  &=&
  h^{-4} (h^3/3-2h^5/5+h^9/7)
  \\
  &=& {\cal O}(h^{-1}).
\end{eqnarray*}
Ups, even the contribution of the first cell blows up under mesh refinement,
and we have not even summed up the contributions of the other cells!

It turns out, that the other cells are not really problematic (since the
gradient is bounded there by a constant independent of $h$), but we cannot
really avoid the trouble with the first cell: if instead of the interpolant we
choose some other finite element function that is closer on average to
$x^{1/3}$ than the interpolant above, then we have to increase the slope of
this function, since we have to obey the boundary condition at the left
end. But then we are hit by the weight $(u')^6$. This weight is simply too
strong!

On the other hand, the interpolation of the linear function $\varphi(x)=x$
connecting the boundary values has the finite energy $I(i_h\varphi)=1/10$,
independent of the mesh size. Thus, $i_hx^{1/3}$ cannot be the minimizer of the
energy as $h\rightarrow 0$. This is also easy to see by noting that 
the minimal value of $I$ cannot increase under mesh
refinement: if it is finite for some function on some mesh, then it must be
smaller or equal to that value on a finer mesh, since the original function is
still in the space spanned by the shape functions on the finer grid, as finite
element spaces are nested. However, the computation above shows that we should
not be surprised if the value of the functional does not converge to zero, but
rather some finite value.

There is one more conclusion to be drawn from the blow-up lesson above: we
cannot expect the finite dimensional approximation to be close to the root
function at the left end of the domain, for any mesh we choose! Because, if it
would, then its energy would have to blow up. And we will see exactly this
in the results section below.


\section{What to do?}

After this somewhat theoretical introduction, let us just once in our life
have fun with pure mathematics, and actually see what happens in this problem
when we run the finite element method on it. So here it goes: to find the
minimum of $I(u)$, we have to find its stationary point. The condition for
this reads
\begin{equation*}
  I'(u,\varphi) 
  = 
  \int_0^1 6 (x-u^3) (u')^5 \{ (x-u^3)\varphi' - u^2 u' \varphi\}\ dx,
\end{equation*}
for all test functions $\varphi$ from the same space as that from which we
take $u$, but with zero boundary conditions. If this space allows us to
integrate by parts, then we could associate this with a two point boundary
value problem
\begin{equation}
  \label{eq:equation}
  -(x-u^3) u^2(u')^6
  - \frac{d}{dx} \left\{(x-u^3)^2 (u')^5\right\} = 0,
  \qquad\qquad u(0)=0,
  \quad u(1)=1.
\end{equation}
Note that this equation degenerates wherever $u^3=x$, which is at least the
case at $x\in\{0,1\}$ due to the prescribed boundary values for $u$, but
possibly at other places as well.  However, for finite elements, we will want
to have the equation in weak form anyway.  Since the equation is still
nonlinear, one may be tempted to compute iterates
$u_{k+1}=u_k+\alpha_k\delta u_k$ using a Newton method for updates $\delta
u_k$, like in
\begin{equation*}
  I''(u_k,\delta u_k,\varphi) 
  = 
  -I'(u_k, \varphi).
\end{equation*}
However, since $I''(u_k,\cdot,\cdot)$ may be an indefinite operator (and, as
numerical experiments indicate, is in fact during typical computations), we
don't want to use this. Instead, we use a gradient method, for which we
compute updates according to the following scheme:
\begin{equation}
  \left<\delta u_k,\varphi\right>
  = 
  -I'(u_k, \varphi).
\end{equation}
For the scalar product on the left hand side, there are multiple valid ways;
we choose the mesh dependent definition $\left<u,v\right> = \int_\Omega (uv +
h(x)^2 \nabla u\cdot \nabla v)\; dx$, where the weight $h(x)^2$, i.e. using
the local mesh width, is chosen so that the definition is dimensionally
consistent. It also yields a matrix on the left hand side that is simple to
invert, as it is the sum of the well-conditioned mass matrix, and a Laplace
matrix times a factor that counters the growth of condition number of the
Laplace matrix.

The step length $\alpha_k$ is then computed using a one-dimensional line search
finding
\begin{equation}
  \label{eq:linesearch}
  \alpha_k = \arg\min_\alpha I(u_k+\alpha\delta u_k),
\end{equation}
or at least an approximation to this using a one-dimensional Newton method
which itself has a line search. The details of this can be found in the code.
We iterate the updates and line searches until the change in energy $I(u_k)$
becomes too small to warrant any further iterations.

The basic idea that you should get in all this is that we formulate the
optimization method in a function space, and will only discretize each step
separately. A number of subsequent steps will be done on the same mesh, before
we refine it and go on to do the same on the next finer mesh.

As for mesh refinement, it is instructional to recall how residual based error
estimates like the one used in the Kelly et al.~error estimator are usually
derived (the Kelly estimator is the one that we have used in most of the
previous example programs). In a similar way, by looking at the residual of
the strong form of the nonlinear equation we attempt here to solve, see
equation \eqref{eq:equation}, we may be tempted to consider the following
expression for refinement of cell $K$:
\begin{eqnarray}
  \label{eq:error-estimate}
  \eta_K^2 &=&
  h^2 \left\|
    (x-u_h^3) (u_h')^4 \left\{ u_h^2 (u_h')^2 + 5(x-u_h^3)u_h'' + 2u_h'(1-3u_h^2u_h') \right\}
  \right\|^2_K
  \notag \\
  && +
  h \left| (x-u_h^3)^2 [(u_h')^5] \right|^2_{\partial K},
\end{eqnarray}
where $[\cdot]$ is the jump of a quantity across an intercell boundary, and 
$|\cdot|_{\partial K}$ is the sum of the quantity evaluated at the two end
points of a cell. Note that in the evaluation of the jump, we have made use of
the fact that $x-u_h^3$ is a continuous quantity, and can therefore be taken
out of the jump operator.

All these details actually matter -- while writing the program I have played
around with many settings and different versions of the code, and the result
is that if you don't have a good line search, good stopping criteria, the
right metric (scalar product) for the gradient method, good initial values,
and a good refinement criterion, then the nonlinear solver gets stuck quite
readily for this highly nonlinear problem. Initially, I was hardly able to
find solutions for which the energy dropped below 0.005, while the energy
after the final iteration of the program as it is is usually around 0.0003,
and occasionally down to less than 3e-5.

However, this is not enough. In the program, we start the solver on the coarse
mesh many times, with randomly perturbed starting values, and while it
converges it yields a different solution, with a different energy every
time. One can therefore not say that the solver converges to a certain energy,
and we can't answer the question what the smallest value of $I(u)$ might be in
$W^{1,\infty}$. This is unsatisfactory, but maybe to be expected for such a
contrived and pathological problem. Consider it an example in programming with
deal.II then, and not an example in solving this particular problem.


\section{Implementation}

The program implements all the steps mentioned above, and we will discuss them
in the commented code below. In general, however, note that formulating the
Newton method in function spaces, and only discretizing afterwards has
consequences: we have to linearize around $u_k$ when we want to compute
$\delta u_k$, and we have to sum up these two functions afterwards.  However,
they may be living on different grids, if we have refined the grid before this
step, so we will have to present a way to actually get a function from one
grid to another. The \textrm{SolutionTransfer} class will help us here. On the
other hand, discretizing every nonlinear step separately has the advantage
that we can do the initial steps, when we are still far away from the
solution, on a coarse mesh, and only go on to more expensive computations when
we home in on an solution. We will use a
very simplistic strategy for when we refine the mesh (every fifth nonlinear
step), though. Realistic programs solving nonlinear problems will have to be more
clever in this respect, but it suffices for the purposes of this program.

We will show some of the things that are really simple in 1d (but sometimes
different from what we are used to in 2d or 3d). Apart from this, the program
does not contain much new stuff, but if it explains a few of the techniques
that are available for nonlinear problems and in particular 1d problems, then
this is not so bad, after all.

\textbf{Note:} As shown below, the program starts the nonlinear solver from 10 different
initial values, and outputs the results. This is not actually too many, but we
did so to keep run-time short (around 1:30 minutes on my laptop). If you want to
increase the number of realizations, you may want to switch to optimized mode
(by setting the ``debug-mode'' flag in the Makefile to ``off''), and increase
the number of realizations to a larger value. On the same machine as above, I
can compute 100 realizations in optimized mode in about 2 minutes. For
this particular program, the difference between debug and optimized mode is
thus about a factor of 7-8, which can be explained by the fact that we ask the
compiler to do optimizations on the code only in the latter mode, but in most
part due to the fact that in optimized mode all the ``Assert'' checks are
thrown out that make sure that function arguments are correct, and that check
the internal consistency of the library. The library contains several
thousands of these checks, and they significantly slow down debug
computations, but we feel that the benefit of finding programming errors
earlier and including where the problem exactly appeared to be of significantly
greater value than faster run-time. After all, all production runs of programs
should be done in optimized mode anyway.

A slowdown of a factor of 7-8 is unusual, however. For 2d and 3d applications,
a typical value is around 4. 

\end{document}
