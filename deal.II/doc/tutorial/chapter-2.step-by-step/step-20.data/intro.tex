\documentclass{article}
\usepackage{amsmath}
\renewcommand{\vec}[1]{\mathbf{#1}}
\begin{document}

This program is devoted to two aspects: the use of mixed finite elements -- in
particular Raviart-Thomas elements -- and using block matrices to define
solvers and preconditioners that use the substructure of the system
matrix. The equation we are going to solve is again the Laplace equation,
though with a matrix-valued coefficient:
\begin{align*}
  -\nabla \cdot K(\vec x) \nabla p &= f \qquad && \text{in $\Omega$}, \\
  p &= g && \text{on $\partial\Omega$}.
\end{align*}
$K(\vec x)$ is assumed to be uniformly positive definite, i.e. there is
$\alpha>0$ such that the eigenvalues $\lambda_i(\vec x)$ of $K(x)$ satisfy
$\lambda_i(\vec x)\ge \alpha$. The use of the symbol $p$ instead of the usual
$u$ for the solution variable will become clear in the next section.

After discussing the equation and the formulation we are going to use to solve
it, this introduction will cover the use of block matrices and vectors, the
definition of solvers and preconditioners, and finally the actual testcase we
are going to solve.

\subsection*{Formulation, weak form, and discrete problem}

In the form above, the Laplace equation is considered a good model equation
for fluid flow in porous media. In particular, if flow is so slow that all
dynamic effects such as the acceleration terms in the Navier-Stokes equation
become irrelevant, and if the flow pattern is stationary, then the Laplace
equation models the pressure that drives the flow reasonable well. Because the
solution variable is a pressure, we here use the name $p$ instead of the
name $u$ more commonly used for the solution of partial differential equations.

Typical applications of this view of the Laplace equation are then modeling
groundwater flow, or the flow of hydrocarbons in oil reservoirs. In these
applications, $K$ is then the permeability tensor, i.e. a measure for how much
resistence the soil or rock matrix asserts on the fluid flow. In the
applications just named, a desirable feature is that the numerical scheme is
locally conservative, i.e. that whatever flows into a cell also flows out of
it (or the difference is equal to the integral over the source terms over each
cell, if the sources are nonzero). However, as it turns out, the usual
discretizations of the Laplace equation do not satisfy this property. On the
other hand, one can achieve this by choosing a different formulation.

To this end, one first introduces a second variable, called the flux, $\vec
u=-K\nabla p$. By its definition, the flux is a vector in the negative
direction of the pressure gradient, multiplied by the permeability tensor. If
the permeability tensor is proportional to the unit matrix, this equation is
easy to understand and intuitive: the higher the permeability, the higher the
flux; and the flux is proportional to the gradient of the pressure, going from
areas of high pressure to areas of low pressure.

With this second variable, one then finds an alternative version of the
Laplace equation, called the mixed formulation:
\begin{align*}
  K^{-1} \vec u - \nabla p &= 0 \qquad && \text{in $\Omega$}, \\
  -\text{div}\ \vec u &= 0 \qquad && \text{in $\Omega$}, \\
  p &= g \qquad && \text{on $\partial\Omega$}.
\end{align*}

The weak formulation of this problem is found by multiplying the two
equations with test functions and integrating some terms by parts:
\begin{align*}
  A(\{\vec u,p\},\{\vec v,q\}) = F(\{\vec v,q\}),
\end{align*}
where
\begin{align*}
  A(\{\vec u,p\},\{\vec v,q\})
  &=
  (\vec v, K^{-1}\vec u)_\Omega - (\text{div}\ \vec v, p)_\Omega
  - (q,\text{div}\ \vec u)_\Omega
  \\
  F(\{\vec v,q\}) &= -(g,\vec v\cdot \vec n)_{\partial\Omega} - (f,q)_\Omega.
\end{align*}
Here, $\vec n$ is the outward normal vector at the boundary. Note how in this
formulation, Dirichlet boundary values of the original problem are
incorporated in the weak form.

To be well-posed, we have to look for solutions and test functions in the
space $H(\text{div})=\{\vec w\in L^2(\Omega)^d:\ \text{div}\ \vec w\in L^2\}$
for $\vec u,\vec v$, and $L^2$ for $p,q$. It is a well-known fact stated in
almost every book on finite element theory that if one chooses discrete finite
element spaces for the approximation of $\vec u,p$ inappropriately, then the
resulting discrete saddle-point problem is instable and the discrete solution
will not converge to the exact solution.

To overcome this, a number of different finite element pairs for $\vec u,p$
have been developed that lead to a stable discrete problem. One such pair is
to use the Raviart-Thomas spaces $RT(k)$ for the velocity $\vec u$ and
discontinuous elements of class $DQ(k)$ for the pressure $p$. For details
about these spaces, we refer in particular to the book on mixed finite element
methods by Brezzi and Fortin, but many other books on the theory of finite
elements, for example the classic book by Brenner and Scott, also state the
relevant results.


\subsection*{Assembling the linear system}

The deal.II library (of course) implements Raviart-Thomas elements $RT(k)$ of
arbitrary order $k$, as well as discontinuous elements $DG(k)$. If we forget
about their particular properties for a second, we then have to solve a
discrete problem
\begin{align*}
  A(x_h,w_h) = F(w_h),
\end{align*}
with the bilinear form and right hand side as stated above, and $x_h=\{\vec
u_h,p_h\}$, $w_h=\{\vec v_h,q_h\}$. Both $x_h$ and $w_h$ are from the space
$X_h=RT(k)\times DQ(k)$, where $RT(k)$ is itself a space of $dim$-dimensional
functions to accomodate for the fact that the flow velocity is vector-valued.
The necessary question then is: how do we do this in a program?

Vector-valued elements have already been discussed in previous tutorial
programs, the first time and in detail in step-8. The main difference there
was that the vector-valued space $V_h$ is uniform in all its components: the
$dim$ components of the displacement vector are all equal and from the same
function space. What we could therefore do was to build $V_h$ as the outer
product of the $dim$ times the usual $Q(1)$ finite element space, and by this
make sure that all our shape functions have only a single non-zero vector
component. Instead of dealing with vector-valued shape functions, all we did
in step-8 was therefore to look at the (scalar) only non-zero component and
use the \texttt{fe.system\_to\_component\_index(i).first} call to figure out
which component this actually is.

This doesn't work with Raviart-Thomas elements: following from their
construction to satisfy certain regularity properties of the space
$H(\text{div})$, the shape functions of $RT(k)$ are usually nonzero in all
their vector components at once. For this reason, were
\texttt{fe.system\_to\_component\_index(i).first} applied to determine the only
nonzero component of shape function $i$, an exception would be generated. What
we really need to do is to get at \textit{all} vector components of a shape
function. In deal.II diction, we call such finite elements
\textit{non-primitive}, whereas finite elements that are either scalar or for
which every vector-valued shape function is nonzero only in a single vector
component are called \textit{primitive}.

So what do we have to do for non-primitive elements? To figure this out, let
us go back in the tutorial programs, almost to the very beginnings. There, we
learned that we use the \texttt{FEValues} class to determine the values and
gradients of shape functions at quadrature points. For example, we would call
\texttt{fe\_values.shape\_value(i,q\_point)} to obtain the value of the
\texttt{i}th shape function on the quadrature point with number
\texttt{q\_point}. Later, in step-8 and other tutorial programs, we learned
that this function call also works for vector-valued shape functions (of
primitive finite elements), and that it returned the value of the only
non-zero component of shape function \texttt{i} at quadrature point
\texttt{q\_point}.

For non-primitive shape functions, this is clearly not going to work: there is
no single non-zero vector component of shape function \texttt{i}, and the call 
to \texttt{fe\_values.shape\_value(i,q\_point)} would consequently not make
much sense. However, deal.II offers a second function call,
\texttt{fe\_values.shape\_value\_component(i,q\_point,comp)} that returns the
value of the \texttt{comp}th vector component of shape function  \texttt{i} at
quadrature point \texttt{q\_point}, where \texttt{comp} is an index between
zero and the number of vector components of the present finite element; for
example, the element we will use to describe velocities and pressures is going
to have $dim+1$ components. It is worth noting that this function call can
also be used for primitive shape functions: it will simply return zero for all
components except one; for non-primitive shape functions, it will in general
return a non-zero value for more than just one component.

We could now attempt to rewrite the bilinear form above in terms of vector
components. For example, in 2d, the first term could be rewritten like this
(note that $u_0=x_0, u_1=x_1, p=x_2$):
\begin{align*}
  (\vec u_h^i, K^{-1}\vec u_h^j)
  =
  &\left((x_h^i)_0, K^{-1}_{00} (x_h^j)_0\right) +
   \left((x_h^i)_0, K^{-1}_{01} (x_h^j)_1\right) + \\
  &\left((x_h^i)_1, K^{-1}_{10} (x_h^j)_0\right) +
   \left((x_h^i)_1, K^{-1}_{11} (x_h^j)_1\right).
\end{align*}
If we implemented this, we would get code like this:
\begin{verbatim}
  for (unsigned int q=0; q<n_q_points; ++q) 
    for (unsigned int i=0; i<dofs_per_cell; ++i)
      for (unsigned int j=0; j<dofs_per_cell; ++j)
        local_matrix(i,j) += (Kinverse[q]][0][0] *
                              fe_values.shape_value_component(i,q,0) *
                              fe_values.shape_value_component(j,q,0) 
                              +
                              Kinverse[q]][0][1] *
                              fe_values.shape_value_component(i,q,0) *
                              fe_values.shape_value_component(j,q,1) 
                              +
                              Kinverse[q]][1][0] *
                              fe_values.shape_value_component(i,q,1) *
                              fe_values.shape_value_component(j,q,0) 
                              +
                              Kinverse[q]][1][1] *
                              fe_values.shape_value_component(i,q,1) *
                              fe_values.shape_value_component(j,q,1) 
                             )
                             *
                             fe_values.JxW(q);
\end{verbatim}
This is, at best, tedious, error prone, and not dimension independent. There
are obvious ways to make things dimension independent, but in the end, the
code is simply not pretty. What would be much nicer is if we could simply
extract the $\vec u$ and $p$ components of a shape function $x_h^i$. In the
program we do that, by writing functions like this one:
\begin{verbatim}
template <int dim>
Tensor<1,dim>
extract_u (const FEValuesBase<dim> &fe_values,
           const unsigned int i,
           const unsigned int q)
{
  Tensor<1,dim> tmp;

  for (unsigned int d=0; d<dim; ++d)
    tmp[d] = fe_values.shape_value_component (i,q,d);

  return tmp;
}
\end{verbatim}

What this function does is, given an \texttt{fe\_values} object, to extract
the values of the first $dim$ components of shape function \texttt{i} at
quadrature points \texttt{q}, that is the velocity components of that shape
function. Put differently, if we write shape functions $x_h^i$ as the tuple
$\{\vec u_h^i,p_h^i\}$, then the function returns the velocity part of this
tuple. Note that the velocity is of course a $dim$-dimensional tensor, and
that the function returns a corresponding object.

Likewise, we have a function that extracts the pressure component of a shape
function:
\begin{verbatim}
template <int dim>
double extract_p (const FEValuesBase<dim> &fe_values,
                  const unsigned int i,
                  const unsigned int q)
{
  return fe_values.shape_value_component (i,q,dim);
}
\end{verbatim}
Finally, the bilinear form contains terms involving the gradients of the
velocity component of shape functions. To be more precise, we are not really
interested in the full gradient, but only the divergence of the velocity
components, i.e. $\text{div}\ \vec u_h^i=\sum_{d=0}^{dim-1} \frac \partial
{\partial x_d} (\vec
u_h^i)_d$. Here's a function that returns this quantity:
\begin{verbatim}
template <int dim>
double
extract_div_u (const FEValuesBase<dim> &fe_values,
               const unsigned int i,
               const unsigned int q)
{
  double divergence = 0;
  for (unsigned int d=0; d<dim; ++d)
    divergence += fe_values.shape_grad_component (i,q,d)[d];

  return divergence;
}
\end{verbatim}

With these three functions, all of which are completely dimension independent
and will therefore also work in 3d, assembling the local matrix and right hand
side contributions becomes a charm:
\begin{verbatim}
for (unsigned int q=0; q<n_q_points; ++q) 
  for (unsigned int i=0; i<dofs_per_cell; ++i)
    {
      const Tensor<1,dim> phi_i_u = extract_u (fe_values, i, q);
      const double div_phi_i_u    = extract_div_u (fe_values, i, q);
      const double phi_i_p        = extract_p (fe_values, i, q);
           
      for (unsigned int j=0; j<dofs_per_cell; ++j)
        {
          const Tensor<1,dim> phi_j_u = extract_u (fe_values, j, q);
          const double div_phi_j_u    = extract_div_u (fe_values, j, q);
          const double phi_j_p        = extract_p (fe_values, j, q);
               
          local_matrix(i,j) += (phi_i_u * Kinverse[q] * phi_j_u
                                - div_phi_i_u * phi_j_p
                                - phi_i_p * div_phi_j_u)
                               * fe_values.JxW(q);
        }

      local_rhs(i) += -(phi_i_p *
                        rhs_values[q] *
                        fe_values.JxW(q));
    }
\end{verbatim}
This very closely resembles the form we have originally written down the
bilinear form and right hand side.

There is one final term that we have to take care of: the right hand side
contained the term $(g,\vec v\cdot \vec n)_{\partial\Omega}$, constituting the
weak enforcement of pressure boundary conditions. We have already seen in
step-7 how to deal with face integrals: essentially exactly the same as with
domain integrals, except that we have to use the \texttt{FEFaceValues} class
instead of \texttt{FEValues}. To compute the boundary term we then simply have
to loop over all boundary faces and integrate there. If you look closely at
the definitions of the \texttt{extract\_*} functions above, you will realize
that it isn't even necessary to write new functions that extract the velocity
and pressure components of shape functions from \texttt{FEFaceValues} objects:
both \texttt{FEValues} and \texttt{FEFaceValues} are derived from a common
base class, \texttt{FEValuesBase}, and the extraction functions above can
therefore deal with both in exactly the same way. Assembling the missing
boundary term then takes on the following form:
\begin{verbatim}
for (unsigned int face_no=0;
     face_no<GeometryInfo<dim>::faces_per_cell;
     ++face_no)
  if (cell->at_boundary(face_no))
    {
      fe_face_values.reinit (cell, face_no);
    
      pressure_boundary_values
        .value_list (fe_face_values.get_quadrature_points(),
                     boundary_values);

      for (unsigned int q=0; q<n_face_q_points; ++q) 
        for (unsigned int i=0; i<dofs_per_cell; ++i)
          {
            const Tensor<1,dim>
              phi_i_u = extract_u (fe_face_values, i, q);
                
            local_rhs(i) += -(phi_i_u *
                              fe_face_values.normal_vector(q) *
                              boundary_values[q] *
                              fe_face_values.JxW(q));
        }
  }
\end{verbatim}

You will find the exact same code as above in the sources for the present
program. We will therefore not comment much on it below.


\subsection*{Linear solvers and preconditioners}



\subsection*{Definition of the testcase}

In this tutorial program, we will solve the Laplace equation in mixed
formulation as stated above. Since we want to monitor convergence of the
solution inside the program, we choose right hand side, boundary conditions,
and the coefficient so that we recover a solution function known to us. In
particular, we choose the pressure solution
\begin{align*}
  p = -\left(\frac \alpha 2 xy^2 + \beta x - \frac \alpha 6 x^2\right),
\end{align*}
and for the coefficient we choose the unit matrix $K_{ij}=\delta_{ij}$ for
simplicity. Consequently, the exact velocity satisfies
\begin{align*}
  \vec u = 
  \begin{pmatrix}
    \frac \alpha 2 y^2 + \beta - \frac \alpha 2 x^2 \\
    \alpha xy
  \end{pmatrix}.
\end{align*}
This solution was chosen since it is exactly divergence free, making it a
realistic testcase for incompressible fluid flow. By consequence, the right
hand side equals $f=0$, and as boundary values we have to choose
$g=p|_{\partial\Omega}$.

For the computations in this program, we choose $\alpha=0.3,\beta=1$. You can
find the resulting solution in the ``Results'' section below, after the
commented program.

\end{document}
