\documentclass{article}
\usepackage{amsmath}
\renewcommand{\vec}[1]{\mathbf{#1}}
\begin{document}

This program is devoted to two aspects: the use of mixed finite elements -- in
particular Raviart-Thomas elements -- and using block matrices to define
solvers, preconditioners, and nested versions of those that use the
substructure of the system matrix. The equation we are going to solve is again
the Laplace equation, though with a matrix-valued coefficient:
\begin{align*}
  -\nabla \cdot K(\vec x) \nabla p &= f \qquad && \text{in $\Omega$}, \\
  p &= g && \text{on $\partial\Omega$}.
\end{align*}
$K(\vec x)$ is assumed to be uniformly positive definite, i.e. there is
$\alpha>0$ such that the eigenvalues $\lambda_i(\vec x)$ of $K(x)$ satisfy
$\lambda_i(\vec x)\ge \alpha$. The use of the symbol $p$ instead of the usual
$u$ for the solution variable will become clear in the next section.

After discussing the equation and the formulation we are going to use to solve
it, this introduction will cover the use of block matrices and vectors, the
definition of solvers and preconditioners, and finally the actual test case we
are going to solve.

\subsection*{Formulation, weak form, and discrete problem}

In the form above, the Laplace equation is considered a good model equation
for fluid flow in porous media. In particular, if flow is so slow that all
dynamic effects such as the acceleration terms in the Navier-Stokes equation
become irrelevant, and if the flow pattern is stationary, then the Laplace
equation models the pressure that drives the flow reasonable well. Because the
solution variable is a pressure, we here use the name $p$ instead of the
name $u$ more commonly used for the solution of partial differential equations.

Typical applications of this view of the Laplace equation are then modeling
groundwater flow, or the flow of hydrocarbons in oil reservoirs. In these
applications, $K$ is then the permeability tensor, i.e. a measure for how much
resistance the soil or rock matrix asserts on the fluid flow. In the
applications just named, a desirable feature is that the numerical scheme is
locally conservative, i.e. that whatever flows into a cell also flows out of
it (or the difference is equal to the integral over the source terms over each
cell, if the sources are nonzero). However, as it turns out, the usual
discretizations of the Laplace equation do not satisfy this property. On the
other hand, one can achieve this by choosing a different formulation.

To this end, one first introduces a second variable, called the flux, $\vec
u=-K\nabla p$. By its definition, the flux is a vector in the negative
direction of the pressure gradient, multiplied by the permeability tensor. If
the permeability tensor is proportional to the unit matrix, this equation is
easy to understand and intuitive: the higher the permeability, the higher the
flux; and the flux is proportional to the gradient of the pressure, going from
areas of high pressure to areas of low pressure.

With this second variable, one then finds an alternative version of the
Laplace equation, called the mixed formulation:
\begin{align*}
  K^{-1} \vec u - \nabla p &= 0 \qquad && \text{in $\Omega$}, \\
  -\text{div}\ \vec u &= 0 \qquad && \text{in $\Omega$}, \\
  p &= g \qquad && \text{on $\partial\Omega$}.
\end{align*}

The weak formulation of this problem is found by multiplying the two
equations with test functions and integrating some terms by parts:
\begin{align*}
  A(\{\vec u,p\},\{\vec v,q\}) = F(\{\vec v,q\}),
\end{align*}
where
\begin{align*}
  A(\{\vec u,p\},\{\vec v,q\})
  &=
  (\vec v, K^{-1}\vec u)_\Omega - (\text{div}\ \vec v, p)_\Omega
  - (q,\text{div}\ \vec u)_\Omega
  \\
  F(\{\vec v,q\}) &= -(g,\vec v\cdot \vec n)_{\partial\Omega} - (f,q)_\Omega.
\end{align*}
Here, $\vec n$ is the outward normal vector at the boundary. Note how in this
formulation, Dirichlet boundary values of the original problem are
incorporated in the weak form.

To be well-posed, we have to look for solutions and test functions in the
space $H(\text{div})=\{\vec w\in L^2(\Omega)^d:\ \text{div}\ \vec w\in L^2\}$
for $\vec u,\vec v$, and $L^2$ for $p,q$. It is a well-known fact stated in
almost every book on finite element theory that if one chooses discrete finite
element spaces for the approximation of $\vec u,p$ inappropriately, then the
resulting discrete saddle-point problem is instable and the discrete solution
will not converge to the exact solution.

To overcome this, a number of different finite element pairs for $\vec u,p$
have been developed that lead to a stable discrete problem. One such pair is
to use the Raviart-Thomas spaces $RT(k)$ for the velocity $\vec u$ and
discontinuous elements of class $DQ(k)$ for the pressure $p$. For details
about these spaces, we refer in particular to the book on mixed finite element
methods by Brezzi and Fortin, but many other books on the theory of finite
elements, for example the classic book by Brenner and Scott, also state the
relevant results.


\subsection*{Assembling the linear system}

The deal.II library (of course) implements Raviart-Thomas elements $RT(k)$ of
arbitrary order $k$, as well as discontinuous elements $DG(k)$. If we forget
about their particular properties for a second, we then have to solve a
discrete problem
\begin{align*}
  A(x_h,w_h) = F(w_h),
\end{align*}
with the bilinear form and right hand side as stated above, and $x_h=\{\vec
u_h,p_h\}$, $w_h=\{\vec v_h,q_h\}$. Both $x_h$ and $w_h$ are from the space
$X_h=RT(k)\times DQ(k)$, where $RT(k)$ is itself a space of $dim$-dimensional
functions to accommodate for the fact that the flow velocity is vector-valued.
The necessary question then is: how do we do this in a program?

Vector-valued elements have already been discussed in previous tutorial
programs, the first time and in detail in step-8. The main difference there
was that the vector-valued space $V_h$ is uniform in all its components: the
$dim$ components of the displacement vector are all equal and from the same
function space. What we could therefore do was to build $V_h$ as the outer
product of the $dim$ times the usual $Q(1)$ finite element space, and by this
make sure that all our shape functions have only a single non-zero vector
component. Instead of dealing with vector-valued shape functions, all we did
in step-8 was therefore to look at the (scalar) only non-zero component and
use the \texttt{fe.system\_to\_component\_index(i).first} call to figure out
which component this actually is.

This doesn't work with Raviart-Thomas elements: following from their
construction to satisfy certain regularity properties of the space
$H(\text{div})$, the shape functions of $RT(k)$ are usually nonzero in all
their vector components at once. For this reason, were
\texttt{fe.system\_to\_component\_index(i).first} applied to determine the only
nonzero component of shape function $i$, an exception would be generated. What
we really need to do is to get at \textit{all} vector components of a shape
function. In deal.II diction, we call such finite elements
\textit{non-primitive}, whereas finite elements that are either scalar or for
which every vector-valued shape function is nonzero only in a single vector
component are called \textit{primitive}.

So what do we have to do for non-primitive elements? To figure this out, let
us go back in the tutorial programs, almost to the very beginnings. There, we
learned that we use the \texttt{FEValues} class to determine the values and
gradients of shape functions at quadrature points. For example, we would call
\texttt{fe\_values.shape\_value(i,q\_point)} to obtain the value of the
\texttt{i}th shape function on the quadrature point with number
\texttt{q\_point}. Later, in step-8 and other tutorial programs, we learned
that this function call also works for vector-valued shape functions (of
primitive finite elements), and that it returned the value of the only
non-zero component of shape function \texttt{i} at quadrature point
\texttt{q\_point}.

For non-primitive shape functions, this is clearly not going to work: there is
no single non-zero vector component of shape function \texttt{i}, and the call 
to \texttt{fe\_values.shape\_value(i,q\_point)} would consequently not make
much sense. However, deal.II offers a second function call,
\texttt{fe\_values.shape\_value\_component(i,q\_point,comp)} that returns the
value of the \texttt{comp}th vector component of shape function  \texttt{i} at
quadrature point \texttt{q\_point}, where \texttt{comp} is an index between
zero and the number of vector components of the present finite element; for
example, the element we will use to describe velocities and pressures is going
to have $dim+1$ components. It is worth noting that this function call can
also be used for primitive shape functions: it will simply return zero for all
components except one; for non-primitive shape functions, it will in general
return a non-zero value for more than just one component.

We could now attempt to rewrite the bilinear form above in terms of vector
components. For example, in 2d, the first term could be rewritten like this
(note that $u_0=x_0, u_1=x_1, p=x_2$):
\begin{align*}
  (\vec u_h^i, K^{-1}\vec u_h^j)
  =
  &\left((x_h^i)_0, K^{-1}_{00} (x_h^j)_0\right) +
   \left((x_h^i)_0, K^{-1}_{01} (x_h^j)_1\right) + \\
  &\left((x_h^i)_1, K^{-1}_{10} (x_h^j)_0\right) +
   \left((x_h^i)_1, K^{-1}_{11} (x_h^j)_1\right).
\end{align*}
If we implemented this, we would get code like this:
\begin{verbatim}
  for (unsigned int q=0; q<n_q_points; ++q) 
    for (unsigned int i=0; i<dofs_per_cell; ++i)
      for (unsigned int j=0; j<dofs_per_cell; ++j)
        local_matrix(i,j) += (k_inverse_values[q][0][0] *
                              fe_values.shape_value_component(i,q,0) *
                              fe_values.shape_value_component(j,q,0) 
                              +
                              k_inverse_values[q][0][1] *
                              fe_values.shape_value_component(i,q,0) *
                              fe_values.shape_value_component(j,q,1) 
                              +
                              k_inverse_values[q][1][0] *
                              fe_values.shape_value_component(i,q,1) *
                              fe_values.shape_value_component(j,q,0) 
                              +
                              k_inverse_values[q][1][1] *
                              fe_values.shape_value_component(i,q,1) *
                              fe_values.shape_value_component(j,q,1) 
                             )
                             *
                             fe_values.JxW(q);
\end{verbatim}
This is, at best, tedious, error prone, and not dimension independent. There
are obvious ways to make things dimension independent, but in the end, the
code is simply not pretty. What would be much nicer is if we could simply
extract the $\vec u$ and $p$ components of a shape function $x_h^i$. In the
program we do that, by writing functions like this one:
\begin{verbatim}
template <int dim>
Tensor<1,dim>
extract_u (const FEValuesBase<dim> &fe_values,
           const unsigned int i,
           const unsigned int q)
{
  Tensor<1,dim> tmp;

  for (unsigned int d=0; d<dim; ++d)
    tmp[d] = fe_values.shape_value_component (i,q,d);

  return tmp;
}
\end{verbatim}

What this function does is, given an \texttt{fe\_values} object, to extract
the values of the first $dim$ components of shape function \texttt{i} at
quadrature points \texttt{q}, that is the velocity components of that shape
function. Put differently, if we write shape functions $x_h^i$ as the tuple
$\{\vec u_h^i,p_h^i\}$, then the function returns the velocity part of this
tuple. Note that the velocity is of course a $dim$-dimensional tensor, and
that the function returns a corresponding object.

Likewise, we have a function that extracts the pressure component of a shape
function:
\begin{verbatim}
template <int dim>
double extract_p (const FEValuesBase<dim> &fe_values,
                  const unsigned int i,
                  const unsigned int q)
{
  return fe_values.shape_value_component (i,q,dim);
}
\end{verbatim}
Finally, the bilinear form contains terms involving the gradients of the
velocity component of shape functions. To be more precise, we are not really
interested in the full gradient, but only the divergence of the velocity
components, i.e. $\text{div}\ \vec u_h^i = \sum_{d=0}^{dim-1}
\frac{\partial}{\partial x_d} (\vec u_h^i)_d$. Here's a function that returns
this quantity:
\begin{verbatim}
template <int dim>
double
extract_div_u (const FEValuesBase<dim> &fe_values,
               const unsigned int i,
               const unsigned int q)
{
  double divergence = 0;
  for (unsigned int d=0; d<dim; ++d)
    divergence += fe_values.shape_grad_component (i,q,d)[d];

  return divergence;
}
\end{verbatim}

With these three functions, all of which are completely dimension independent
and will therefore also work in 3d, assembling the local matrix and right hand
side contributions becomes a charm:
\begin{verbatim}
for (unsigned int q=0; q<n_q_points; ++q) 
  for (unsigned int i=0; i<dofs_per_cell; ++i)
    {
      const Tensor<1,dim> phi_i_u = extract_u (fe_values, i, q);
      const double div_phi_i_u    = extract_div_u (fe_values, i, q);
      const double phi_i_p        = extract_p (fe_values, i, q);
           
      for (unsigned int j=0; j<dofs_per_cell; ++j)
        {
          const Tensor<1,dim> phi_j_u = extract_u (fe_values, j, q);
          const double div_phi_j_u    = extract_div_u (fe_values, j, q);
          const double phi_j_p        = extract_p (fe_values, j, q);
               
          local_matrix(i,j) += (phi_i_u * k_inverse_values[q] * phi_j_u
                                - div_phi_i_u * phi_j_p
                                - phi_i_p * div_phi_j_u)
                               * fe_values.JxW(q);
        }

      local_rhs(i) += -(phi_i_p *
                        rhs_values[q] *
                        fe_values.JxW(q));
    }
\end{verbatim}
This very closely resembles the form we have originally written down the
bilinear form and right hand side.

There is one final term that we have to take care of: the right hand side
contained the term $(g,\vec v\cdot \vec n)_{\partial\Omega}$, constituting the
weak enforcement of pressure boundary conditions. We have already seen in
step-7 how to deal with face integrals: essentially exactly the same as with
domain integrals, except that we have to use the \texttt{FEFaceValues} class
instead of \texttt{FEValues}. To compute the boundary term we then simply have
to loop over all boundary faces and integrate there. If you look closely at
the definitions of the \texttt{extract\_*} functions above, you will realize
that it isn't even necessary to write new functions that extract the velocity
and pressure components of shape functions from \texttt{FEFaceValues} objects:
both \texttt{FEValues} and \texttt{FEFaceValues} are derived from a common
base class, \texttt{FEValuesBase}, and the extraction functions above can
therefore deal with both in exactly the same way. Assembling the missing
boundary term then takes on the following form:
\begin{verbatim}
for (unsigned int face_no=0;
     face_no<GeometryInfo<dim>::faces_per_cell;
     ++face_no)
  if (cell->at_boundary(face_no))
    {
      fe_face_values.reinit (cell, face_no);
    
      pressure_boundary_values
        .value_list (fe_face_values.get_quadrature_points(),
                     boundary_values);

      for (unsigned int q=0; q<n_face_q_points; ++q) 
        for (unsigned int i=0; i<dofs_per_cell; ++i)
          {
            const Tensor<1,dim>
              phi_i_u = extract_u (fe_face_values, i, q);
                
            local_rhs(i) += -(phi_i_u *
                              fe_face_values.normal_vector(q) *
                              boundary_values[q] *
                              fe_face_values.JxW(q));
        }
  }
\end{verbatim}

You will find the exact same code as above in the sources for the present
program. We will therefore not comment much on it below.


\subsection*{Linear solvers and preconditioners}

After assembling the linear system we are faced with the task of solving
it. The problem here is: the matrix has a zero block at the bottom right
(there is no term in the bilinear form that couples the pressure $p$ with the
pressure test function $q$), and it is indefinite. At least it is
symmetric. In other words: the Conjugate Gradient method is not going to
work. We would have to resort to other iterative solvers instead, such as
MinRes, SymmLQ, or GMRES, that can deal with indefinite systems. However, then
the next problem immediately surfaces: due to the zero block, there are zeros
on the diagonal and none of the usual preconditioners (Jacobi, SSOR) will work
as they require division by diagonal elements.


\subsubsection*{Solving using the Schur complement}

In view of this, let us take another look at the matrix. If we sort our
degrees of freedom so that all velocity come before all pressure variables,
then we can subdivide the linear system $AX=B$ into the following blocks:
\begin{align*}
  \begin{pmatrix}
    M & B^T \\ B & 0
  \end{pmatrix}
  \begin{pmatrix}
    U \\ P
  \end{pmatrix}
  =
  \begin{pmatrix}
    F \\ G
  \end{pmatrix},
\end{align*}
where $U,P$ are the values of velocity and pressure degrees of freedom,
respectively, $M$ is the mass matrix on the velocity space, $B$ corresponds to
the negative divergence operator, and $B^T$ is its transpose and corresponds
to the negative gradient.

By block elimination, we can then re-order this system in the following way
(multiply the first row of the system by $BM^{-1}$ and then subtract the
second row from it):
\begin{align*}
  BM^{-1}B^T P &= BM^{-1} F - G, \\
  MU &= F - B^TP.
\end{align*}
Here, the matrix $S=BM^{-1}B^T$ (called the \textit{Schur complement} of $A$)
is obviously symmetric and, owing to the positive definiteness of $M$ and the
fact that $B^T$ has full column rank, $S$ is also positive
definite. 

Consequently, if we could compute $S$, we could apply the Conjugate Gradient
method to it. However, computing $S$ is expensive, and $S$ is most
likely also a full matrix. On the other hand, the CG algorithm doesn't require
us to actually have a representation of $S$, it is sufficient to form
matrix-vector products with it. We can do so in steps: to compute $Sv$, we 
\begin{itemize}
\item form $w = B^T v$;
\item solve $My = w$ for $y=M^{-1}w$, using the CG method applied to the
  positive definite and symmetric mass matrix $M$;
\item form $z=By$ to obtain $Sv=z$.
\end{itemize}
We will implement a class that does that in the program. Before showing its
code, let us first note that we need to multiply with $M^{-1}$ in several
places here: in multiplying with the Schur complement $S$, forming the right
hand side of the first equation, and solving in the second equation. From a
coding viewpoint, it is therefore appropriate to relegate such a recurring
operation to a class of its own. We call it \texttt{InverseMatrix}. As far as
linear solvers are concerned, this class will have all operations that solvers
need, which in fact includes only the ability to perform matrix-vector
products; we form them by using a CG solve (this of course requires that the
matrix passed to this class satisfies the requirements of the CG
solvers). Here are the relevant parts of the code that implements this:
\begin{verbatim}
class InverseMatrix
{
  public:
    InverseMatrix (const SparseMatrix<double> &m);

    void vmult (Vector<double>       &dst,
                const Vector<double> &src) const;

  private:
    const SmartPointer<const SparseMatrix<double> > matrix;
    // ...
};


void InverseMatrix::vmult (Vector<double>       &dst,
                           const Vector<double> &src) const
{
  SolverControl solver_control (src.size(), 1e-8*src.l2_norm());
  SolverCG<>    cg (solver_control, vector_memory);

  cg.solve (*matrix, dst, src, PreconditionIdentity());        
}
\end{verbatim}
Once created, objects of this class can act as matrices: they perform
matrix-vector multiplications. How this is actually done is irrelevant to the
outside world.

Using this class, we can then write a class that implements the Schur
complement in much the same way: to act as a matrix, it only needs to offer a
function to perform a matrix-vector multiplication, using the algorithm
above. Here are again the relevant parts of the code:
\begin{verbatim}
class SchurComplement 
{
  public:
    SchurComplement (const BlockSparseMatrix<double> &A,
                     const InverseMatrix             &Minv);

    void vmult (Vector<double>       &dst,
                const Vector<double> &src) const;

  private:
    const SmartPointer<const BlockSparseMatrix<double> > system_matrix;
    const SmartPointer<const InverseMatrix>              m_inverse;
    
    mutable Vector<double> tmp1, tmp2;
};


void SchurComplement::vmult (Vector<double>       &dst,
                             const Vector<double> &src) const
{
  system_matrix->block(0,1).vmult (tmp1, src);
  m_inverse->vmult (tmp2, tmp1);
  system_matrix->block(1,0).vmult (dst, tmp2);
}
\end{verbatim}

In this code, the constructor takes a reference to a block sparse matrix for
the entire system, and a reference to an object representing the inverse of
the mass matrix. It stores these using \texttt{SmartPointer} objects (see
step-7), and additionally allocates two temporary vectors \texttt{tmp1} and
\texttt{tmp2} for the vectors labeled $w,y$ in the list above.

In the matrix-vector multiplication function, the product $Sv$ is performed in
exactly the order outlined above. Note how we access the blocks $B^T$ and $B$
by calling \texttt{system\_matrix->block(0,1)} and
\texttt{system\_matrix->block(1,0)} respectively, thereby picking out
individual blocks of the block system. Multiplication by $M^{-1}$ happens
using the object introduced above.

With all this, we can go ahead and write down the solver we are going to
use. Essentially, all we need to do is form the right hand sides of the two
equations defining $P$ and $U$, and then solve them with the Schur complement
matrix and the mass matrix, respectively:
\begin{verbatim}
template <int dim>
void MixedLaplaceProblem<dim>::solve () 
{
  const InverseMatrix m_inverse (system_matrix.block(0,0));
  Vector<double> tmp (solution.block(0).size());
  
  {
    Vector<double> schur_rhs (solution.block(1).size());

    m_inverse.vmult (tmp, system_rhs.block(0));
    system_matrix.block(1,0).vmult (schur_rhs, tmp);
    schur_rhs -= system_rhs.block(1);

    SolverControl solver_control (system_matrix.block(0,0).m(),
                                  1e-6*schur_rhs.l2_norm());
    SolverCG<>    cg (solver_control);

    cg.solve (SchurComplement(system_matrix, m_inverse),
              solution.block(1),
              schur_rhs,
              PreconditionIdentity());
  }
  {
    system_matrix.block(0,1).vmult (tmp, solution.block(1));
    tmp *= -1;
    tmp += system_rhs.block(0);
    
    m_inverse.vmult (solution.block(0), tmp);
  }
}
\end{verbatim}

This code looks more impressive than it actually is. At the beginning, we
declare an object representing $M^{-1}$ and a temporary vector (of the size of
the first block of the solution, i.e. with as many entries as there are
velocity unknowns), and the two blocks surrounded by braces then solve the two
equations for $P$ and $U$, in this order. Most of the code in each of the two
blocks is actually devoted to constructing the proper right hand sides. For
the first equation, this would be $BM^{-1}F-G$, and $-B^TP+G$ for the second
one. The first hand side is then solved with the Schur complement matrix, and
the second simply multiplied with $M^{-1}$. The code as shown uses no
preconditioner (i.e. the identity matrix as preconditioner) for the Schur
complement.



\subsubsection*{A preconditioner for the Schur complement}

One may ask whether it would help if we had a preconditioner for the Schur
complement $S=BM^{-1}B^T$. The general answer, as usual, is: of course. The
problem is only, we don't know anything about this Schur complement matrix. We
do not know its entries, all we know is its action. On the other hand, we have
to realize that our solver is expensive since in each iteration we have to do
one matrix-vector product with the Schur complement, which means that we have
to do invert the mass matrix once in each iteration.

There are different approaches to preconditioning such a matrix. On the one
extreme is to use something that is cheap to apply and therefore has no real
impact on the work done in each iteration. The other extreme is a
preconditioner that is itself very expensive, but in return really brings down
the number of iterations required to solve with $S$. 

We will try something along the second approach, as much to improve the
performance of the program as to demonstrate some techniques. To this end, let
us recall that the ideal preconditioner is, of course, $S^{-1}$, but that is
unattainable. However, how about
\begin{align*}
  \tilde S^{-1} = [B^T (\text{diag}M)^{-1}B]^{-1}
\end{align*}
as a preconditioner? That would mean that every time we have to do one
preconditioning step, we actually have to solve with $\tilde S$. At first,
this looks almost as expensive as solving with $S$ right away. However, note
that in the inner iteration, we do not have to calculate $M^{-1}$, but only
the inverse of its diagonal, which is cheap.

To implement something like this, let us first generalize the
\texttt{InverseMatrix} class so that it can work not only with
\texttt{SparseMatrix} objects, but with any matrix type. This looks like so:
\begin{verbatim}
template <class Matrix>
class InverseMatrix
{
  public:
    InverseMatrix (const Matrix &m);

    void vmult (Vector<double>       &dst,
                const Vector<double> &src) const;

  private:
    const SmartPointer<const Matrix> matrix;

    //...
};


template <class Matrix>
void InverseMatrix<Matrix>::vmult (Vector<double>       &dst,
                                   const Vector<double> &src) const
{
  SolverControl solver_control (src.size(), 1e-8*src.l2_norm());
  SolverCG<> cg (solver_control, vector_memory);

  dst = 0;
  
  cg.solve (*matrix, dst, src, PreconditionIdentity());        
}
\end{verbatim}
Essentially, the only change we have made is the introduction of a template
argument that generalizes the use of \texttt{SparseMatrix}.

The next step is to define a class that represents the approximate Schur
complement. This should look very much like the Schur complement class itself,
except that it doesn't need the object representing $M^{-1}$ any more:
\begin{verbatim}
class ApproximateSchurComplement : public Subscriptor
{
  public:
    ApproximateSchurComplement (const BlockSparseMatrix<double> &A);

    void vmult (Vector<double>       &dst,
                const Vector<double> &src) const;

  private:
    const SmartPointer<const BlockSparseMatrix<double> > system_matrix;
    
    mutable Vector<double> tmp1, tmp2;
};


void ApproximateSchurComplement::vmult (Vector<double>       &dst,
                                        const Vector<double> &src) const
{
  system_matrix->block(0,1).vmult (tmp1, src);
  system_matrix->block(0,0).precondition_Jacobi (tmp2, tmp1);
  system_matrix->block(1,0).vmult (dst, tmp2);
}
\end{verbatim}
Note how the \texttt{vmult} function differs in simply doing one Jacobi sweep
(i.e. multiplying with the inverses of the diagonal) instead of multiplying
with the full $M^{-1}$.

With all this, we already have the preconditioner: it should be the inverse of
the approximate Schur complement, i.e. we need code like this:
\begin{verbatim}
    ApproximateSchurComplement
      approximate_schur_complement (system_matrix);
      
    InverseMatrix<ApproximateSchurComplement>
      preconditioner (approximate_schur_complement)
\end{verbatim}
That's all!

Taken together, the first block of our \texttt{solve()} function will then
look like this:
\begin{verbatim}
    Vector<double> schur_rhs (solution.block(1).size());

    m_inverse.vmult (tmp, system_rhs.block(0));
    system_matrix.block(1,0).vmult (schur_rhs, tmp);
    schur_rhs -= system_rhs.block(1);

    SchurComplement
      schur_complement (system_matrix, m_inverse);
    
    ApproximateSchurComplement
      approximate_schur_complement (system_matrix);
      
    InverseMatrix<ApproximateSchurComplement>
      preconditioner (approximate_schur_complement);
    
    SolverControl solver_control (system_matrix.block(0,0).m(),
                                  1e-6*schur_rhs.l2_norm());
    SolverCG<>    cg (solver_control);

    cg.solve (schur_complement, solution.block(1), schur_rhs,
              preconditioner);
\end{verbatim}
Note how we pass the so-defined preconditioner to the solver working on the
Schur complement matrix.

Obviously, applying this inverse of the approximate Schur complement is a very
expensive preconditioner, almost as expensive as inverting the Schur
complement itself. We can expect it to significantly reduce the number of
outer iterations required for the Schur complement. In fact it does: in a
typical run on 5 times refined meshes using elements of order 0, the number of
outer iterations drops from 164 to 12. On the other hand, we now have to apply
a very expensive preconditioner 12 times. A better measure is therefore simply
the run-time of the program: on my laptop, it drops from 28 to 23 seconds for
this test case. That doesn't seem too impressive, but the savings become more
pronounced on finer meshes and with elements of higher order. For example, a
six times refined mesh and using elements of order 2 yields an improvement of
318 to 12 outer iterations, at a runtime of 338 seconds to 229 seconds. Not
earth shattering, but significant.


\subsubsection*{A remark on similar functionality in deal.II}

As a final remark about solvers and preconditioners, let us note that a
significant amount of functionality introduced above is actually also present
in the library itself. It probably even is more powerful and general, but we
chose to introduce this material here anyway to demonstrate how to work with
block matrices and to develop solvers and preconditioners, rather than using
black box components from the library.

For those interested in looking up the corresponding library classes: the
\texttt{InverseMatrix} is roughly equivalent to the
\texttt{PreconditionLACSolver} class in the library. Likewise, the Schur
complement class corresponds to the \texttt{SchurMatrix} class.


\subsection*{Definition of the test case}

In this tutorial program, we will solve the Laplace equation in mixed
formulation as stated above. Since we want to monitor convergence of the
solution inside the program, we choose right hand side, boundary conditions,
and the coefficient so that we recover a solution function known to us. In
particular, we choose the pressure solution
\begin{align*}
  p = -\left(\frac \alpha 2 xy^2 + \beta x - \frac \alpha 6 x^2\right),
\end{align*}
and for the coefficient we choose the unit matrix $K_{ij}=\delta_{ij}$ for
simplicity. Consequently, the exact velocity satisfies
\begin{align*}
  \vec u = 
  \begin{pmatrix}
    \frac \alpha 2 y^2 + \beta - \frac \alpha 2 x^2 \\
    \alpha xy
  \end{pmatrix}.
\end{align*}
This solution was chosen since it is exactly divergence free, making it a
realistic test case for incompressible fluid flow. By consequence, the right
hand side equals $f=0$, and as boundary values we have to choose
$g=p|_{\partial\Omega}$.

For the computations in this program, we choose $\alpha=0.3,\beta=1$. You can
find the resulting solution in the ``Results'' section below, after the
commented program.

\end{document}
