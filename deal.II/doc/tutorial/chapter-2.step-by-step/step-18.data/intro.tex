\documentclass{article}
\usepackage{amsmath}
\usepackage{amsfonts}
\renewcommand{\vec}[1]{\mathbf{#1}}
\renewcommand{\div}{\mathrm{div}\ }
\begin{document}

This tutorial program is another one in the series on the elasticity problem
that we have already started with step-8 and step-17. It extends it into two
different directions: first, it solves the quasistatic but time dependent
elasticity problem for large deformations with a Lagrangian mesh movement
approach. Secondly, it shows some more techniques for solving such problems
using parallel processing with PETSc's linear algebra. In addition to this, we
show how to work around the main bottleneck of step-17, namely that we
generated graphical output from only one process, and that this scaled very
badly with larger numbers of processes and on large problems. Finally, a good
number of assorted improvements and techniques are demonstrated that have not
been shown yet in previous programs.

As before in step-17, the program runs just as fine on a single sequential
machine as long as you have PETSc installed. Information on how to tell
deal.II about a PETSc installation on your system can be found in the deal.II
README file, which is linked to from the main documentation page
\texttt{doc/index.html} in your installation of deal.II, or on the deal.II
webpage \texttt{http://www.dealii.org/}.


\subsection*{Quasistatic elastic deformation}

\subsubsection*{Motivation of the model}

In general, time-dependent small elastic deformations are described by the
elastic wave equation
\begin{gather}
  \rho \frac{\partial^2 \vec u}{\partial t^2} 
  + c \frac{\partial \vec u}{\partial t}
  - \div ( C \varepsilon(\vec u)) = \vec f
  \qquad
  \text{in $\Omega$},
\end{gather}
where $\vec u=\vec u (\vec x,t)$ is the deformation of the body, $\rho$
and $c$ the density and attenuation coefficient, and $\vec f$ external forces.
In addition, initial conditions
\begin{align}
  \vec u(\cdot, 0) = \vec u_0(\cdot)
  \qquad
  \text{on $\Omega$},
\end{align}
and Dirichlet (displacement) or Neumann (force) boundary conditions need
to be specified for a unique solution:
\begin{align}
  \vec u(\vec x,t) &= \vec d(\vec x,t)
  \qquad
  &&\text{on $\Gamma_D\subset\partial\Omega$},  
  \\
  \vec n \ C \varepsilon(\vec u(\vec x,t)) &= \vec b(\vec x,t)
  \qquad
  &&\text{on $\Gamma_N=\partial\Omega\backslash\Gamma_D$}.
\end{align}
In above formulation, $\varepsilon(\vec u)= \tfrac 12 (\nabla \vec u + \nabla
\vec u^T)$ is the symmetric gradient of the displacement, also called the
\textit{strain}. $C$ is a tensor of rank 4, called the \textit{stress-strain
  tensor} that contains knowledge of the elastic strength of the material; its
symmetry properties make sure that it maps symmetric tensors of rank 2
(``matrices'' of dimension $d$, where $d$ is the spatial dimensionality) onto
symmetric tensors of same rank. We will comment on the roles of the strain and
stress tensors more below. For the moment it suffices to say that we interpret
the term $\div ( C \varepsilon(\vec u))$ as the vector with components $\tfrac
\partial{\partial x_j} C_{ijkl} \varepsilon(\vec u)_{kl}$, where summation
over indices $j,k,l$ is implied.

The quasistatic limit of this equation is motivated as follows: each small
perturbation of the body, for example by changes in boundary condition or the
forcing function, will result in a corresponding change in the configuration
of the body. In general, this will be in the form of waves radiating away from
the location of the disturbance. Due to the presence of the damping term,
these waves will be attenuated on a time scale of, say, $\tau$. Now, assume
that all changes in external configuration happen on times scales that are
much larger than $\tau$. In that case, the dynamic nature of the change is
unimportant: we can consider the body to always be in static equilibrium,
i.e.~we can assume that at all times the body satisfies
\begin{align}
  - \div ( C \varepsilon(\vec u)) &= \vec f
  &&\text{in $\Omega$},
  \\
  \vec u(\vec x,t) &= \vec d(\vec x,t)
  \qquad
  &&\text{on $\Gamma_D$},
  \\
  \vec n \ C \varepsilon(\vec u(\vec x,t)) &= \vec b(\vec x,t)
  \qquad
  &&\text{on $\Gamma_N$}.
\end{align}
Note that the differential equation does not contain any time derivatives any
more -- all time dependence is introduced through boundary conditions and a
possibly time-varying force function $\vec f(\vec x,t)$.

While these equations are sufficient to describe small deformations, computing
large deformations is a little more complicated. To do so, let us first
introduce a stress variable $\sigma$, and write the differential equations in
terms of the stress:
\begin{align}
  - \div \sigma &= \vec f
  &&\text{in $\Omega(t)$},
  \\
  \vec u(\vec x,t) &= \vec d(\vec x,t)
  \qquad
  &&\text{on $\Gamma_D\subset\partial\Omega(t)$},
  \\
  \vec n \ C \varepsilon(\vec u(\vec x,t)) &= \vec b(\vec x,t)
  \qquad
  &&\text{on $\Gamma_N=\partial\Omega(t)\backslash\Gamma_D$}.
\end{align}
Note that these equations are posed on a domain $\Omega(t)$ that
changes with time, with the boundary moving according to the
displacements $\vec u(\vec x,t)$ of the points on the boundary. To
complete this system, we have to specify the relationship between the
stress and the strain, as follows:
\begin{align}
  \label{eq:stress-strain}
  \dot\sigma = C \varepsilon (\dot{\vec u}),
\end{align}
where a dot indicates a time derivative. Both the stress $\sigma$ and the
strain $\varepsilon(\vec u)$ are symmetric tensors of rank 2.


\subsubsection*{Time discretization}

Numerically, this system is solved as follows: first, we discretize
the time component using a backward Euler scheme. This leads to a
discrete equilibrium of force at time step $n$:
\begin{align}
  -\div \sigma^n &= f^n,
\intertext{where}
  \sigma^n &= \sigma^{n-1} + C \varepsilon (\Delta \vec u^n),
\end{align}
and $\Delta \vec u^n$ the incremental displacement for time step
$n$. This way, if we want to solve for the displacement increment, we
have to solve the following system:
\begin{align}
  - \div  C \varepsilon(\Delta\vec u^n) &= \vec f + \div \sigma^{n-1}
  &&\text{in $\Omega(t_{n-1})$},
  \\
  \Delta \vec u^n(\vec x,t) &= \vec d(\vec x,t_n) - \vec d(\vec x,t_{n-1})
  \qquad
  &&\text{on $\Gamma_D\subset\partial\Omega(t_{n-1})$},
  \\
  \vec n \ C \varepsilon(\Delta \vec u^n(\vec x,t)) &= \vec b(\vec x,t_n)-\vec b(\vec x,t_{n-1})
  \qquad
  &&\text{on $\Gamma_N=\partial\Omega(t_{n-1})\backslash\Gamma_D$}.
\end{align}
The weak form of this set of equations, which as usual is the basis for the
finite element formulation, reads as follows: find $\Delta \vec u^n \in
\{v\in H^1(\Omega(t_{n-1}))^d: v|_{\Gamma_D}=\vec d(\cdot,t_n) - \vec d(\cdot,t_{n-1})\}$
such that
\begin{multline}
  \label{eq:linear-system}
  (C \varepsilon(\Delta\vec u^n), \varepsilon(\varphi) )_{\Omega(t_{n-1})}
  = 
  (\vec f, \varphi)_{\Omega(t_{n-1})}
  -(\sigma^{n-1},\varepsilon(\varphi))_{\Omega(t_{n-1})}
  \\
  +(\vec b(\vec x,t_n)-\vec b(\vec x,t_{n-1}), \varphi)_{\Gamma_N}
  \\
  \forall \varphi \in \{v\in H^1(\Omega(t_{n-1}))^d: v|_{\Gamma_D}=0\}.
\end{multline}
We note that in the program we will always assume that there are no boundary
forces, i.e.~$\vec b = 0$, and that the deformation of the body is driven by
body forces $\vec f$ and prescribed boundary displacements $\vec d$ alone. It
is also worth noting that when integrating by parts, we would get terms of
the form
$(C \varepsilon(\Delta\vec u^n), \nabla \varphi )_{\Omega(t_{n-1})}$,
but that we replace it with the term involving the symmetric gradient
$\varepsilon(\varphi)$ instead of $\nabla\varphi$. Due to the symmetry of $C$
the two terms are equivalent, but the symmetric version avoids a potential for
round-off to render the resulting matrix slightly non-symmetric.

The system at time step $n$, to be solved on the old domain
$\Omega(t_{n-1})$, has exactly the form of a stationary elastic
problem, and is therefore similar to what we have already implemented
in previous example programs. We will therefore not comment on the
space discretization beyond saying that we again use lowest order
continuous finite elements.

There are differences, however:
\begin{enumerate}
  \item We have to move the mesh after each time step, in order to be
  able to solve the next time step on a new domain;

  \item We need to know $\sigma^{n-1}$ to compute the next incremental
  displacement, i.e.~we need to compute it at the end of the time step
  to make sure it is available for the next time step. Essentially,
  the stress variable is our window to the history of deformation of
  the body.
\end{enumerate}
These two operations are done in the functions \texttt{move\_mesh} and
\texttt{update\_\-quadrature\_\-point\_history} in the program. While moving
the mesh is only a technicality, updating the stress is a little more
complicated and will be discussed in the next section.


\subsubsection*{Updating the stress variable}

As indicated above, we need to have the stress variable $\sigma^n$ available
when computing time step $n+1$, and we can compute it using
\begin{gather}
  \label{eq:stress-update}
  \sigma^n = \sigma^{n-1} + C \varepsilon (\Delta \vec u^n).  
\end{gather}
There are, despite the apparent simplicity of this equation, two questions
that we need to discuss. The first concerns the way we store $\sigma^n$: even
if we compute the incremental updates $\Delta\vec u^n$ using lowest-order
finite elements, then its symmetric gradient $\varepsilon(\Delta\vec u^n)$ is
in general still a function that is not easy to describe. In particular, it is
not a piecewise constant function, and on general meshes (with cells that are
not rectangles parallel to the coordinate axes) or with non-constant
stress-strain tensors $C$ it is not even a bi- or trilinear function. Thus, it
is a priori not clear how to store $\sigma^n$ in a computer program.

To decide this, we have to see where it is used. The only place where we
require the stress is in the term
$(\sigma^{n-1},\varepsilon(\varphi))_{\Omega(t_{n-1})}$. In practice, we of
course replace this term by numerical quadrature
\begin{gather}
  (\sigma^{n-1},\varepsilon(\varphi))_{\Omega(t_{n-1})}
  =
  \sum_{K\subset {\mathbb{T}}}
  (\sigma^{n-1},\varepsilon(\varphi))_K
  \approx
  \sum_{K\subset {\mathbb{T}}}
  \sum_q
  w_q \ \sigma^{n-1}(\vec x_q) : \varepsilon(\varphi(\vec x_q)),
\end{gather}
where $w_q$ are the quadrature weights and $\vec x_q$ the quadrature points on
cell $K$. This should make clear that what we really need is not the stress
$\sigma^{n-1}$ in itself, but only the values of the stress in the quadrature
points on all cells. This, however, is a simpler task: we only have to provide
a data structure that is able to hold one symmetric tensor of rank 2 for each
quadrature point on all cells (or, since we compute in parallel, all
quadrature points of all cells that the present MPI process ``owns''). At the
end of each time step we then only have to evaluate $\varepsilon(\Delta \vec
u^n(\vec x_q))$, multiply it by the stress-strain tensor $C$, and use the
result to update the stress $\sigma^n(\vec x_q)$ at quadrature point $q$.

The second complication is not visible in our notation as chosen above. It is
due to the fact that we compute $\Delta u^n$ on the domain $\Omega(t_{n-1})$,
and then use this displacement increment to both update the stress as well as
move the mesh nodes around to get to $\Omega(t_n)$ on which the next increment
is computed. What we have to make sure, in this context, is that moving the
mesh does not only involve moving around the nodes, but also making
corresponding changes to the stress variable: the updated stress is a variable
that is defined with respect to the coordinate system of the old mesh, and has
to be transferred to the new mesh. While the updating procedure has already
taken care of the case where the material is compressed or dilated, it has to
be explicitly extended to account for the case that a cell is rotated. To this
end, we have to define a rotation matrix $R(\Delta \vec u^n)$ that describes,
in each point the rotation due to the displacement increments. It is not hard
to see that the actual dependence of $R$ on $\Delta \vec u^n$ can only be
through the curl of the displacement, rather than the displacement itself or
its full gradient (the constant components of the increment describe
translations, its divergence the dilational modes, and the curl the rotational
modes). Since the exact form of $R$ is cumbersome, we only state it in the
program code, and note that the correct updating formula for the stress
variable is then
\begin{gather}
  \label{eq:stress-update+rot}
  \sigma^n
  = 
  R(\Delta \vec u^n)^T 
  [\sigma^{n-1} + C \varepsilon (\Delta \vec u^n)]
  R(\Delta \vec u^n).
\end{gather}
This is all implemented in the function
\texttt{update\_\-quadrature\_\-point\_history} of the example program.


\subsection*{Parallel graphical output}

In the step-17 example program, the main bottleneck for parallel computations
was that only the first processor generated output for the entire domain.
Since generating graphical output is expensive, this did not scale well when
large numbers of processors were involved. However, no viable ways around this
problem were implemented in the library at the time, and the problem was
deferred to a later version.

This functionality has been implemented in the meantime, and this is the time
to explain its use. Basically, what we need to do is let every process
generate graphical output for that subset of cells that it owns, write them
into separate files and have a way to merge them later on. At this point, it
should be noted that none of the graphical output formats known to the author
of this program allows for a simple way to later re-read it and merge it with
other files corresponding to the same simulation. What deal.II therefore
offers is the following: When you call the \texttt{DataOut::build\_patches}
function, an intermediate format is generated that contains all the
information for the data on each cell. Usually, this intermediate format is
then further processed and converted into one of the graphical formats that we
can presently write, such as gmv, eps, ucd, gnuplot, or a number of other
ones. Once written in these formats, there is no way to reconstruct the
necessary information to merge multiple blocks of output. However, the base
classes of \texttt{DataOut} also allows to simply dump the intermediate format to a
file, from which it can later be recovered without loss of information.

This has two advantages: first, simulations may just dump the intermediate
format data during run-time, and the user may later decide which particular
graphics format she wants to have. This way, she does not have to re-run the
entire simulation if graphical output is requested in a different format. One
typical case is that one would like to take a quick look at the data with
gnuplot, and then create high-quality pictures using GMV or OpenDX. Since both
can be generated out of the intermediate format without problem, there is no
need to re-run the simulation.

In the present context, of more interest is the fact that in contrast to any
of the other formats, it is simple to merge multiple files of intermediate
format, if they belong to the same simulation. This is what we will do here:
we will generate one output file in intermediate format for each processor
that belongs to this computation (in the sequential case, this will simply be
a single file). They may then later be read in and merged so that we can
output a single file in whatever graphical format is requested.

The way to do this is to first instruct the \texttt{DataOutBase} class to
write intermediate format rather than in gmv or any other graphical
format. This is simple: just use
\texttt{data\_out.write\_deal\_II\_intermediate}. This will generate one file
called \texttt{solution-TTTT.TTTT.d2} if there is only one processor, or
files \texttt{solution-TTTT.TTTT.NNN.d2} if this is really a parallel
job. Here, \texttt{TTTT.TTTT} denotes the time for which this output has
been generated, and \texttt{NNN} the number of the MPI process that did this.

The next step is to convert this file or these files into whatever
format you like. The program that does this is the step-19 tutorial program:
for example, for the first time step, call it through
\begin{center}
  \texttt{../step-19/step-19 solution-0001.0000.*.d2 solution-0001.0000.gmv}
\end{center}
to merge all the intermediate format files into a single file in GMV
format. More details on the parameters of this program and what it can do for
you can be found in the documentation of the step-19 tutorial program.



\subsection*{Overall structure of the program}

The overall structure of the program can be inferred from the \texttt{run()}
function that first calls \texttt{do\_initial\_timestep()} for the first time
step, and then \texttt{do\_timestep()} on all subsequent time steps. The
difference between these functions is only that on the first time step we
start on a coarse mesh, solve on it, refine the mesh adaptively, and then
start again with a clean state on that new mesh. This procedure gives us a
better starting mesh, although we should of course keep adapting the mesh as
iterations proceed -- this isn't done in this program, but commented on below.

The common part of the two functions treating time steps is that the following
sequence of operations on the present mesh:
\begin{itemize}
\item \texttt{assemble\_system ()} [via \texttt{solve\_timestep ()}]:
  This first function is also the most interesting one. It assembles the
  linear system corresponding to the discretized version of equation
  \eqref{eq:linear-system}. This leads to a system matrix $A_{ij} = \sum_K
  A^K_{ij}$ built up of local contributions on each cell $K$ with entries
  \begin{gather}
    A^K_{ij} = (C \varepsilon(\varphi_i), \varepsilon(\varphi_j))_K;
  \end{gather}
  In practice, $A^K$ is computed using numerical quadrature according to the
  formula
  \begin{gather}
    A^K_{ij} = \sum_q w_q  \varepsilon(\varphi_i(\vec x_q)) : C :
                           \varepsilon(\varphi_j(\vec x_q)),
  \end{gather}
  with quadrature points $\vec x_q$ and weights $w_q$. We have built these
  contributions before, in step-8 and step-17, but in both of these cases we
  have done so rather clumsily by using knowledge of how the rank-4 tensor $C$
  is composed, and considering individual elements of the strain tensors
  $\varepsilon(\varphi_i),\varepsilon(\varphi_j)$. This is not really
  convenient, in particular if we want to consider more complicated elasticity
  models than the isotropic case for which $C$ had the convenient form
  $c_{ijkl}  = \lambda \delta_{ij} \delta_{kl} + \mu (\delta_{ik} \delta_{jl}
  + \delta_{il} \delta_{jk})$. While we in fact do not use a more complicated
  form that this in the present program, we nevertheless want to write it in a
  way that would easily allow for this. It is then natural to introduce
  classes that represent symmetric tensors of rank 2 (for the strains and
  stresses) and 4 (for the stress-strain tensor $C$). Fortunately, deal.II
  provides these: the \texttt{SymmetricTensor<rank,dim>} class template
  provides a full-fledged implementation of such tensors of rank \texttt{rank}
  (which needs to be an even number) and dimension \texttt{dim}.

  What we then need is two things: a way to create the stress-strain rank-4
  tensor $C$ as well as to create a symmetric tensor of rank 2 (the strain
  tensor) from the gradients of a shape function $\varphi_i$ at a quadrature
  point $\vec x_q$ on a given cell. At the top of the implementation of this
  example program, you will find such functions. The first one,
  \texttt{get\_stress\_strain\_tensor}, takes two arguments corresponding to
  the Lam'e constants $\lambda$ and $\mu$ and returns the stress-strain tensor
  for the isotropic case corresponding to these constants (in the program, we
  will choose constants corresponding to steel); it would be simple to replace
  this function by one that computes this tensor for the anisotropic case, or
  taking into account crystal symmetries, for example. The second one,
  \texttt{get\_strain} takes an object of type \texttt{FEValues} and indices
  $i$ and $q$ and returns the symmetric gradient, i.e. the strain,
  corresponding to shape function $\varphi_i(\vec x_q)$, evaluated on the cell
  on which the \texttt{FEValues} object was last reinitialized.

  Given this, the innermost loop of \texttt{assemble\_system} computes the
  local contributions to the matrix in the following elegant way (the variable
  \texttt{stress\_strain\_tensor}, corresponding to the tensor $C$, has
  previously been initialized with the result of the first function above):
  \begin{verbatim}
for (unsigned int i=0; i<dofs_per_cell; ++i)
  for (unsigned int j=0; j<dofs_per_cell; ++j) 
    for (unsigned int q_point=0; q_point<n_q_points;
         ++q_point)
      {
        const SymmetricTensor<2,dim>
          eps_phi_i = get_strain (fe_values, i, q_point),
          eps_phi_j = get_strain (fe_values, j, q_point);

        cell_matrix(i,j) 
          += (eps_phi_i * stress_strain_tensor * eps_phi_j
              *
              fe_values.JxW (q_point));
      }
  \end{verbatim}
  It is worth noting the expressive power of this piece of code, and to
  compare it with the complications we had to go through in previous examples
  for the elasticity problem. (To be fair, the \texttt{SymmetricTensor} class
  template did not exist when these previous examples were written.) For
  simplicity, \texttt{operator*} provides for the (double summation) product
  between symmetric tensors of even rank here.

  Assembling the local contributions 
  \begin{gather}
    \begin{split}
      f^K_i &= 
      (\vec f, \varphi_i)_K -(\sigma^{n-1},\varepsilon(\varphi_i))_K
      \\
      &\approx
      \sum_q
      w_q \left\{
        \vec f(\vec x_q) \cdot \varphi_i(\vec x_q) -
        \sigma^{n-1}_q : \varepsilon(\varphi_i(\vec x_q))
      \right\}
    \end{split}
  \end{gather}
  to the right hand side of \eqref{eq:linear-system} is equally
  straightforward (note that we do not consider any boundary tractions $\vec
  b$ here). Remember that we only had to store the old stress in the
  quadrature points of cells. In the program, we will provide a variable
  \texttt{local\_quadrature\_points\_data} that allows to access the stress
  $\sigma^{n-1}_q$ in each quadrature point. With this the code for the right
  hand side looks as this, again rather elegant:
  \begin{verbatim}
for (unsigned int i=0; i<dofs_per_cell; ++i)
  {
    const unsigned int 
      component_i = fe.system_to_component_index(i).first;

    for (unsigned int q_point=0; q_point<n_q_points; ++q_point)
      {
        const SymmetricTensor<2,dim> &old_stress
          = local_quadrature_points_data[q_point].old_stress;
        
        cell_rhs(i) += (body_force_values[q_point](component_i) *
                        fe_values.shape_value (i,q_point)
                        -
                        old_stress *
                        get_strain (fe_values,i,q_point))
                       *
                       fe_values.JxW (q_point);
      }
  }
  \end{verbatim}
  Note that in the multiplication $\vec f(\vec x_q) \cdot \varphi_i(\vec
  x_q)$, we have made use that for the chosen finite element, only one vector
  component (namely \texttt{component\_i}) of $\varphi_i$ is nonzero, and that
  we therefore also have to consider only one component of $\vec f(\vec
  x_q)$.

  This essentially concludes the new material we present in this function. It
  later has to deal with boundary conditions as well as hanging node
  constraints, but this parallels what we had to do previously in other
  programs already.

\item \texttt{solve\_linear\_problem ()} [via \texttt{solve\_timestep ()}]:
  Unlike the previous one, this function is not really interesting, since it
  does what similar functions have done in all previous tutorial programs --
  solving the linear system using the CG method. It is virtually unchanged
  from step-17.

\item \texttt{update\_quadrature\_point\_history ()} [via
  \texttt{solve\_timestep ()}]: Based on the displacement field $\Delta \vec
  u^n$ computed before, we update the stress values in all quadrature points
  according to \eqref{eq:stress-update} and \eqref{eq:stress-update+rot},
  including the rotation of the coordinate system.

\item \texttt{move\_mesh ()}: Given the solution computed before, in this
  function we deform the mesh by moving each vertex by the displacement vector
  field evaluated at this particular vertex.

\item \texttt{output\_results ()}: This function simply outputs the solution
  based on what we have said above, i.e. every processor computes output only
  for its own portion of the domain, and this can then be later merged by an
  external program. In addition to the solution, we also compute the norm of
  the stress averaged over all the quadrature points on each cell.
\end{itemize}

With this general structure of the code, we only have to define what case
we want to solve. For the present program, we have chosen to simulate the
quasistatic deformation of a vertical cylinder for which the bottom boundary
is fixed and the top boundary is pushed down at a prescribed vertical
velocity. However, the horizontal velocity of the top boundary is left
unspecified -- one can imagine this situation is a well-greased plate pushing
from the top onto the cylinder, the points on the top boundary of the cylinder
being allowed to slide along the surface of the plate, but forced to move
downward by the plate. The inner and outer boundaries of the cylinder are free
and not subject to any prescribed deflection or traction.

The program text will reveal more about how to implement this situation, and
the results section will show what displacement pattern comes out of this
simulation. 

\subsection*{Possible directions for extensions}

The program as is does not really solve an equation that has many applications
in practice: quasi-static material deformation based on a purely elastic law
is almost boring. However, the program may serve as the starting point for
more interesting experiments, and that indeed was the initial motivation for
writing it. Here are some suggestions of what the program is missing and in
what direction it may be extended:

\paragraph*{Plasticity models.} The most obvious extension is to use a more
realistic material model for large-scale quasistatic deformation. The natural
choice for this would be plasticity, in which a nonlinear relationship between
stress and strain replaces equation \eqref{eq:stress-strain}. Plasticity
models are usually rather complicated to program since this dependence is
generally non-smooth. The material can be thought of being able to withstand
only a maximal stress (the yield stress) after which it starts to ``flow''. A
mathematical description to this can be given in the form of a variational
inequality, which alternatively can be treated as minimizing the elastic
energy
\begin{gather}
  E(\vec u) = 
  (\varepsilon(\vec u), C\varepsilon(\vec u))_{\Omega}
  - (\vec f, \vec u)_{\Omega} - (\vec b, \vec u)_{\Gamma_N},
\end{gather}
subject to the constraint
\begin{gather}
  f(\sigma(\vec u)) \le 0
\end{gather}
on the stress. This extension makes the problem to be solved in each time step
nonlinear, so we need another loop within each time step.

Without going into further details of this model, we refer to the excellent
book by Simo and Hughes on ``Computational Inelasticity'' for a
comprehensive overview of computational strategies for solving plastic
models. Alternative, a brief but concise description of an algorithm for
plasticity is given in an article by S. Commend, A. Truty, and Th. Zimmermann,
titled ``Stabilized finite elements applied to 
elastoplasticity: I. Mixed displacement-pressure formulation''
(Computer Methods in Applied Mechanics and Engineering, vol. 193,
pp. 3559--3586, 2004).


\paragraph*{Stabilization issues.} The formulation we have chosen, i.e. using
piecewise (bi-, tri-)linear elements for all components of the displacement
vector, and treating the stress as a variable dependent on the displacement is
appropriate for most materials. However, this so-called displacement-based
formulation becomes unstable and exhibits spurious modes for incompressible or
nearly-incompressible materials. While fluids are usually not elastic (in most
cases, the stress depends on velocity gradients, not displacement gradients,
although there are exceptions such as electro-rheologic fluids), there are a
few solids that are nearly incompressible, for example rubber. Another case is
that many plasticity models ultimately let the material become incompressible,
although this is outside the scope of the present program.

Incompressibility is characterized by Poisson's ratio
\begin{gather*}
  \nu = \frac{\lambda}{2(\lambda+\mu)},
\end{gather*}
where $\lambda,\mu$ are the Lam'e constants of the material.
Physical constraints indicate that $-1\le \nu\le \tfrac 12$. If $\nu$
approaches $\tfrac 12$, then the material becomes incompressible. In that
case, pure displacement-based formulations are no longer appropriate for the
solution of such problems, and stabilization techniques have to be employed
for a stable and accurate solution. The book and paper cited above give
indications as to how to do this, but there is also a large volume of
literature on this subject; a good start to get an overview of the topic can
be found in the references of the paper by
H.-Y. Duan and Q. Lin on ``Mixed finite elements of least-squares type for
elasticity'' (Computer Methods in Applied Mechanics and Engineering, vol. 194,
pp. 1093--1112, 2005).


\paragraph*{Refinement during timesteps.} In the present form, the program
only refines the initial mesh a number of times, but then never again. For any
kind of realistic simulation, one would want to extend this so that the mesh
is refined and coarsened every few time steps instead. This is not hard to do,
in fact, but has been left for future tutorial programs or as an exercise, if
you wish. The main complication one has to overcome is that one has to
transfer the data that is stored in the quadrature points of the cells of the
old mesh only the new mesh, preferably by some sort of projection scheme. This
is slightly messy in the sequential case. However, it becomes complicated once
we run the program in parallel, since then each process only stores this data
for the cells it owned on the old mesh, and it may need to know the values of
the quadrature point data on other cells if the corresponding cells on the new
mesh are assigned to this process after subdividing the new mesh. A global
communication of these data elements is therefore necessary, making the entire
process a little more unpleasant.


\paragraph*{Ensuring mesh regularity.} At present, the program makes no attempt
to make sure that a cell, after moving its vertices at the end of the time
step, still has a valid geometry (i.e. that its Jacobian determinant is
positive and bounded away from zero everywhere). It is, in fact, not very hard
to set boundary values and forcing terms in such a way that one gets distorted
and inverted cells rather quickly. Certainly, in some cases of large
deformation, this is unavoidable with a mesh of finite mesh size, but in some
other cases this should be preventable by appropriate mesh refinement and/or a
reduction of the time step size. The program does not do that, but a more
sophisticated version definitely should employ some sort of heuristic defining
what amount of deformation of cells acceptable, and what isn't.


\subsection*{Compiling the program}

Finally, just to remind everyone: the program runs in 3d (see the definition
of the \texttt{elastic\_problem} variable in \texttt{main()}, unlike almost all
of the other example programs. While the compiler doesn't care what dimension
it compiles for, the linker has to know which library to link with. And as
explained in other places, this requires slight changes to the Makefile
compared to the other tutorial programs. In particular, everywhere where the
2d versions of libraries are mentioned, one needs to change this to
3d. Conversely, if you want to run the program in 2d (after making the
necessary changes to accommodate for a 2d geometry), you have to change the
Makefile back to allow for 2d.

\end{document}
