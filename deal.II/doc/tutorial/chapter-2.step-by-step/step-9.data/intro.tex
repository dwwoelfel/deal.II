%%%%%%%%%%%%%%%%%%%%%%%%%%%%%%%%%%%%%
%% intro.html was generated from this file 
%% with latex2html and some hand work
%% (copying out the relevant parts from the 
%% generated html file, replacing IMG=
%% by the proper path)
%%%%%%%%%%%%%%%%%%%%%%%%%%%%%%%%%%%%%
\documentclass{article}
\usepackage{amsmath}
\usepackage{amsfonts}

\renewcommand{\vec}[1]{{\mathbf #1}}
\renewcommand{\div}{\nabla \cdot}
\begin{document}

In this example, our aims are the following:
\begin{itemize}
  \item solve the advection equation $\beta \cdot \nabla u = f$;
  \item show how we can use multiple threads to get quicker to
    the desired results if we have a multi-processor machine;
  \item develop a simple refinement criterion.
\end{itemize}
While the second aim is difficult to describe in general terms without
reference to the code, we will discuss the other two aims in the
following. The use of multiple threads will then be detailed at the
relevant places within the program. Furthermore, there exists a report on this
subject, which is also available online from the ``Documentation'' section of
the deal.II homepage.

\paragraph{Discretizing the advection equation.}
In the present example program, we shall numerically approximate the
solution of the advection equation
$$
  \beta \cdot \nabla u = f,
$$
where $\beta$ is a vector field that describes advection direction and
speed (which may be dependent on the space variables), $f$ is a source
function, and $u$ is the solution. The physical process that this
equation describes is that of a given flow field $\beta$, with which
another substance is transported, the density or concentration of
which is given by $u$. The equation does not contain diffusion of this
second species within its carrier substance, but there are source
terms.

It is obvious that at the inflow, the above equation needs to be
augmented by boundary conditions:
$$
  u = g \qquad\qquad \text{on $\partial\Omega_-$},
$$
where $\partial\Omega_-$ describes the inflow portion of the boundary and is
formally defined by
$$
  \partial\Omega_- 
  =
  \{\vec x\in \partial\Omega: \beta\cdot\vec n(\vec x) < 0\},
$$
and $\vec n(\vec x)$ being the outward normal to the domain at point
$\vec x\in\partial\Omega$. This definition is quite intuitive, since
as $\vec n$ points outward, the scalar product with $\beta$ can only
be negative if the transport direction $\beta$ points inward, i.e. at
the inflow boundary. The mathematical theory states that we must not
pose any boundary condition on the outflow part of the boundary.

As it is stated, the transport equation is not stably solvable using
the standard finite element method, however. The problem is that
solutions to this equation possess only insufficient regularity
orthogonal to the transport direction: while they are smooth parallel
to $\beta$, they may be discontinuous perpendicular to this
direction. These discontinuities lead to numerical instabilities that
make a stable solution by a straight-forward discretization
impossible. We will thus use the streamline diffusion stabilized
formulation, in which we test the equation with test functions $v +
\delta \beta\cdot\nabla v$ instead of $v$, where $\delta$ is a
parameter that is chosen in the range of the (local) mesh width $h$;
good results are usually obtained by setting $\delta=0.1h$. Note that
the modification in the test function vanishes as the mesh size tends
to zero. We will not discuss reasons, pros, and cons of the streamline
diffusion method, but rather use it ``as is'', and refer the
interested reader to the sufficiently available literature; every
recent good book on finite elements should have a discussion of that
topic.

Using the test functions as defined above, the weak formulation of
our stabilized problem reads: find a discrete function $u_h$ such that
for all discrete test functions $v_h$ there holds
$$
  (\beta \cdot \nabla u_h, v_h + \delta \beta\cdot\nabla v_h)_\Omega
  -
  (\beta\cdot \vec n u_h, v_h)_{\partial\Omega_-}
  =
  (f, v_h + \delta \beta\cdot\nabla v_h)_\Omega
  -
  (\beta\cdot \vec n g, v_h)_{\partial\Omega_-}.
$$
Note that we have included the inflow boundary values into the weak
form, and that the respective terms to the left hand side operator are
positive definite due to the fact that $\beta\cdot\vec n<0$ on the
inflow boundary. One would think that this leads to a system matrix
to be inverted of the form
$$
  a_{ij} =
  (\beta \cdot \nabla \varphi_i,
   \varphi_j + \delta \beta\cdot\nabla \varphi_j)_\Omega
  -
  (\beta\cdot \vec n \varphi_i, \varphi_j)_{\partial\Omega_-},
$$
with basis functions $\varphi_i,\varphi_j$.  However, this is a
pitfall that happens to every numerical analyst at least once
(including the author): we have here expanded the solution
$u_h = u_i \varphi_i$, but if we do so, we will have to solve the 
problem 
$$
  \vec u^T A = \vec f^T,
$$
where $\vec u=(u_i)$, i.e. we have to solve the transpose problem of
what we might have expected naively. In order to obtain the usual form
of the linear system, it is therefore best to rewrite the weak
formulation to
$$
  (v_h + \delta \beta\cdot\nabla v_h, \beta \cdot \nabla u_h)_\Omega
  -
  (\beta\cdot \vec n v_h, u_h)_{\partial\Omega_-}
  =
  (v_h + \delta \beta\cdot\nabla v_h, f)_\Omega
  -
  (\beta\cdot \vec n v_h, g)_{\partial\Omega_-}
$$
and then to obtain
$$
  a_{ij} =
  (\varphi_i + \delta \beta \cdot \nabla \varphi_i,
   \beta\cdot\nabla \varphi_j)_\Omega
  -
  (\beta\cdot \vec n \varphi_i, \varphi_j)_{\partial\Omega_-},
$$
as system matrix. We will assemble this matrix in the program.

There remains the solution of this linear system of equations. As the
resulting matrix is no more symmetric positive definite, we can't
employ the usual CG method any more. Suitable for the solution of
systems as the one at hand is the BiCGStab (bi-conjugate gradients
stabilized) method, which is also available in deal.II, so we will use
it.


Regarding the exact form of the problem which we will solve, we use
the following domain and functions (in $d=2$ space dimensions):
\begin{eqnarray*}
  \Omega &=& [-1,1]^d \\
  \beta(\vec x) 
  &=&
  \left(
    \begin{array}{c}1 \\ 1+\frac 45 \sin(8\pi x)\end{array} 
  \right),
  \\
  f(\vec x)
  &=&
  \left\{
    \begin{array}{ll}
        \frac 1{10 s^d} & 
        \text{for $|\vec x-\vec x_0|<s$}, \\
        0 & \text{else},
    \end{array}
  \right.
  \qquad\qquad
  \vec x_0 
  =
  \left(
    \begin{array}{c} -\frac 34 \\ -\frac 34\end{array}
  \right),
  \\
  g
  &=&
  e^{5(1-|\vec x|^2)} \sin(16\pi|\vec x|^2).
\end{eqnarray*}
For $d>2$, we extend $\beta$ and $\vec x_0$ by the same as the last
component. Regarding these functions, we have the following
annotations:
\begin{itemize}
\item The advection field $\beta$ transports the solution roughly in
diagonal direction from lower left to upper right, but with a wiggle
structure superimposed.
\item The right hand side adds to the field generated by the inflow
boundary conditions a bulb in the lower left corner, which is then
transported along.
\item The inflow boundary conditions impose a weighted sinusoidal
structure that is transported along with the flow field. Since $|\vec
x|\ge 1$ on the boundary, the weighting term never gets very large.
\end{itemize}


\paragraph{A simple refinement criterion.}
In all previous examples with adaptive refinement, we have used an
error estimator first developed by Kelly et al., which assigns to each
cell $K$ the following indicator:
$$
  \eta_K =
  \left(
    \frac {h_K}{12}
    \int_{\partial K}
      [\partial_n u_h]^2 \; d\sigma
  \right)^{1/2},
$$
where $[\partial n u_h]$ denotes the jump of the normal derivatives
across a face $\gamma\subset\partial K$ of the cell $K$. It can be
shown that this error indicator uses a discrete analogue of the second
derivatives, weighted by a power of the cell size that is adjusted to
the linear elements assumed to be in use here:
$$
  \eta_K \approx 
  C h \| \nabla^2 u \|_K,
$$
which itself is related to the error size in the energy norm.

The problem with this error indicator in the present case is that it
assumes that the exact solution possesses second derivatives. This is
already questionable for solutions to Laplace's problem in some cases,
although there most problems allow solutions in $H^2$. If solutions
are only in $H^1$, then the second derivatives would be singular in
some parts (of lower dimension) of the domain and the error indicators
would not reduce there under mesh refinement. Thus, the algorithm
would continuously refine the cells around these parts, i.e. would
refine into points or lines (in 2d).

However, for the present case, solutions are usually not even in $H^1$
(and this missing regularity is not the exceptional case as for
Laplace's equation), so the error indicator described above is not
really applicable. We will thus develop an indicator that is based on
a discrete approximation of the gradient. Although the gradient often
does not exist, this is the only criterion available to us, at least
as long as we use continuous elements as in the present
example. To start with, we note that given two cells $K$, $K'$ of
which the centers are connected by the vector $\vec y_{KK'}$, we can
approximate the directional derivative of a function $u$ as follows:
$$
  \frac{\vec y_{KK'}^T}{|\vec y_{KK'}|} \nabla u
  \approx
  \frac{u(K') - u(K)}{|\vec y_{KK'}|},
$$
where $u(K)$ and $u(K')$ denote $u$ evaluated at the centers of the
respective cells. We now multiply the above approximation by 
$\vec y_{KK'}/|\vec y_{KK'}|$ and sum over all neighbors $K'$ of $K$:
$$
  \underbrace{
    \left(\sum_{K'} \frac{\vec y_{KK'} \vec y_{KK'}^T}
                         {|\vec y_{KK'}|^2}\right)}_{=:Y}
  \nabla u
  \approx
  \sum_{K'}
  \frac{\vec y_{KK'}}{|\vec y_{KK'}|}
  \frac{u(K') - u(K)}{|\vec y_{KK'}|}.
$$
If the vectors $\vec y_{KK'}$ connecting $K$ with its neighbors span
the whole space (i.e. roughly: $K$ has neighbors in all directions),
then the term in parentheses in the left hand side expression forms a
regular matrix, which we can invert to obtain an approximation of the
gradient of $u$ on $K$:
$$
  \nabla u
  \approx
  Y^{-1}
  \left(
    \sum_{K'}
    \frac{\vec y_{KK'}}{|\vec y_{KK'}|}
    \frac{u(K') - u(K)}{|\vec y_{KK'}|}
  \right).
$$
We will denote the approximation on the right hand side by
$\nabla_h u(K)$, and we will use the following quantity as refinement
criterion:
$$
  \eta_K = h^{1+d/2} |\nabla_h u_h(K)|,
$$
which is inspired by the following (not rigorous) argument:
\begin{eqnarray*}
  \|u-u_h\|^2_{L_2} 
  &\le&
  C h^2 \|\nabla u\|^2_{L_2}
\\
  &\approx&
  C
  \sum_K
  h_K^2 \|\nabla u\|^2_{L_2(K)}
\\
  &\le&
  C
  \sum_K
  h_K^2 h_K^d \|\nabla u\|^2_{L_\infty(K)}
\\
  &\approx&
  C
  \sum_K
  h_K^{2+d} |\nabla_h u_h(K)|^2
\end{eqnarray*}

\end{document}
