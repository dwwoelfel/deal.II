\documentclass{article}

\begin{document}

In this program, we will mainly consider two aspects:
\begin{itemize}
\item Verification of correctness of the program and generation of convergence
tables;
\item Non-homogeneous Neumann boundary conditions for the Helmholtz equation.
\end{itemize}
Besides these topics, again a variety of improvements and tricks will be
shown. 

\paragraph{Verification of correctness.} There has probably never been a
non-trivial finite element program that worked right from the start. It is
therefore necessary to find ways to verify whether a computed solution is
correct or not. Usually, this is done by choosing the set-up of a simulation
such that we know the exact continuous solution and evaluate the difference
between continuous and computed discrete solution. If this difference
converges to zero with the right order of convergence, this is already a good
indication of correctness, although there may be other sources of error
persisting which have only a small contribution to the total error or are of
higher order.

In this example, we will not go into the theories of systematic software
verification which is a very complicated problem. Rather we will demonstrate
the tools which deal.II can offer in this respect. This is basically centered
around the functionality of a single function, \texttt{integrate\_difference}.
This function computes the difference between a given continuous function and
a finite element field in various norms on each cell. At present, the
supported norms are the following, where $u$ denotes the continuous function
and $u_h$ the finite element field, and $K$ is an element of the
triangulation:
\begin{eqnarray*}
  {\| u-u_h \|}_{L_1(K)} &=& \int_K |u-u_h| \; dx,
  \\
  {\| u-u_h \|}_{L_2(K)} &=& \left( \int_K |u-u_h|^2 \; dx \right)^{1/2},
  \\
  {\| u-u_h \|}_{L_\infty(K)} &=& \max_{x  \in K} |u(x) - u_h(x)|,
  \\
  {| u-u_h |}_{H^1(K)} &=& \left( \int_K |\nabla(u-u_h)|^2 \; dx \right)^{1/2},
  \\
  {\| u-u_h \|}_{H^1(K)} &=& \left( {\| u-u_h \|}^2_{L_2(K)} 
                                   +{| u-u_h |}^2_{H^1(K)}    \right)^{1/2}.
\end{eqnarray*}
All these norms and semi-norms can also be evaluated with weighting functions,
for example in order to exclude singularities from the determination of the
global error. The function also works for vector-valued functions.  It should
be noted that all these quantities are evaluated using quadrature formulas;
the choice of the right quadrature formula is therefore crucial to the
accurate evaluation of the error. This holds in particular for the $L_\infty$
norm, where we evaluate the maximal deviation of numerical and exact solution
only at the quadrature points; one should then not try to use a quadrature
rule with points only at points where super-convergence might occur.

The function \texttt{integrate\_difference} evaluates the desired norm on each
cell $K$ of the triangulation and returns a vector which holds these
values for each cell. From the local values, we can then obtain the global error. For
example, if the vector $(e_i)$ contains the local $L_2$ norms, then
$$
  E = \| {\mathbf e} \| = \left( \sum_i e_i^2 \right)^{1/2}
$$
is the global $L_2$ error.

In the program, we will show how to evaluate and use these quantities, and we
will monitor their values under mesh refinement. Of course, we have to choose
the problem at hand such that we can explicitly state the solution and its
derivatives, but since we want to evaluate the correctness of the program,
this is only reasonable. If we know that the program produces the correct
solution for one (or, if one wants to be really sure: many) specifically
chosen right hand sides, we can be rather confident that it will also compute
the correct solution for problems where we don't know the exact values.

In addition to simply computing these quantities, we will show how to generate
nicely formatted tables from the data generated by this program that
automatically computes convergence rates etc. In addition, we will compare
different strategies for mesh refinement.


\paragraph{Non-homogeneous Neumann boundary conditions.} The second, totally
unrelated, subject of this example program is the use of non-homogeneous
boundary conditions. These are included into the variational form using
boundary integrals which we have to evaluate numerically when assembling the
right hand side vector.

Before we go into programming, let's have a brief look at the mathematical
formulation. The equation which we want to solve is Helmholtz's equation
``with the nice sign'':
$$
  -\Delta u + u = f,
$$
on the square $[-1,1]^2$, augmented by boundary conditions
$$
  u = g_1
$$
on some part $\Gamma_1$ of the boundary $\Gamma$, and
$$
  {\mathbf n}\cdot \nabla u = g_2
$$
on the rest $\Gamma_2 = \Gamma \backslash \Gamma_1$.

We choose the right hand side function $f$ such that the exact solution is
$$
  u(x) = \sum_{i=1}^3 \exp\left(-\frac{|x-x_i|^2}{\sigma^2}\right)
$$
where the centers $x_i$ of the exponentials are 
  $x_1=(-\frac 12,\frac 12)$,
  $x_2=(-\frac 12,-\frac 12)$, and
  $x_3=(\frac 12,-\frac 12)$.
The half width is set to $\sigma=\frac 13$.

We further choose $\Gamma_1=\Gamma \cap\{\{x=1\} \cup \{y=1\}\}$, and there
set $g_1$ such that it resembles the exact values of $u$. Likewise, we choose
$g_2$ on the remaining portion of the boundary to be the exact normal
derivatives of the continuous solution.

Using the above definitions, we can state the weak formulation of the
equation, which reads: find $u\in H^1_g=\{v\in H^1: v|_{\Gamma_1}=g_1\}$ such
that
$$
  {(\nabla u, \nabla v)}_\Omega + {(u,v)}_\Omega
  =
  {(f,v)}_\Omega + {(g_2,v)}_{\Gamma_2}
$$
for all test functions $v\in H^1_0=\{v\in H^1: v|_{\Gamma_1}=0\}$. The
boundary term ${(g_2,v)}_{\Gamma_2}$ has appeared by integration by parts and
using $\partial_n u=g$ on $\Gamma_2$ and $v=0$ on $\Gamma_1$. The cell
matrices and vectors which we use to build the global matrices and right hand
side vectors in the discrete formulation therefore look like this:
\begin{eqnarray*}
  A_{ij}^K &=& \left(\nabla \varphi_i, \nabla \varphi_j\right)_K
              +\left(\varphi_i, \varphi_j\right)_K,
  \\
  f_i^K &=& \left(f,\varphi_i\right)_K
           +\left(g_2, \varphi_i\right)_{\partial K\cap \Gamma_2}.
\end{eqnarray*}
Since the generation of the domain integrals has been shown in previous
examples several times, only the generation of the contour integral is of
interest here. It basically works along the following lines: for domain
integrals we have the \texttt{FEValues} class that provides values and
gradients of the shape values, as well as Jacobian determinants and other
information and specified quadrature points in the cell; likewise, there is a
class \texttt{FEFaceValues} that performs these tasks for integrations on
faces of cells. One provides it with a quadrature formula for a manifold with
dimension one less than the dimension of the domain is, and the cell and the
number of its face on which we want to perform the integration. The class will
then compute the values, gradients, normal vectors, weights, etc. at the
quadrature points on this face, which we can then use in the same way as for
the domain integrals. The details of how this is done are shown in the
following program.


\end{document}

%%% Local Variables: 
%%% mode: latex
%%% TeX-master: t
%%% End: 
